\section{Total Energy for GFN2-xTB}
\begin{equation}
\begin{split}
\E{GFN2-xTB} &= \E[0]{rep}+\E[0,1,2]{disp}+\E[1]{EHT}+\E[2]{IES+IXC}+\E[2]{AES+AXC}+\E[3]{IES+IXC}\\
&=\E{rep}+\E{disp}^{D4'}+\E{EHT}+\E{\gamma}+\E{AES}+\E{AXC}+\E{\Gamma}^{GFN2}
\end{split}
\end{equation}
\subsection{Repulsion Energy}
\begin{align}
\E{rep} &= \frac{1}{2}\sum_{A,B}\frac{Z^{eff}_A Z^{eff}_B}{R_{AB}}e^{-\sqrt{a_Aa_B}(R_{AB})^{(k_f)}}\\
k_f &= \begin{cases}1 & if A,B\in\{\text{H},\text{He}\}\\\frac{3}{2}&otherwise\end{cases} 
\end{align}
$Z^{eff}$ and $a$ are variables fitted for each element. A,B are the labels of atoms. 
Since we only have C and H in our systems we can simplify this quite a bit in code. 
$R_{AB}$ is the distance between the A and B atoms.
\subsection{Extended Hückel Theory Energy}
\begin{align}
    \E{EHT} &= \sum_{\mu\nu}P_{\mu\nu}H^{EHT}_{\mu\nu}\\
    P_{\mu\nu} &= P^{(0)}_{\mu\nu}+ \delta P_{\mu\nu}\\
    P^{(0)}_{\mu\nu}&=\sum_j^{occ. MO}c_{\mu j}c_{j\nu}\\ 
    \delta P_{\mu\nu} &=??\quad\text{comes from the iteration, can be skipped for now}
\end{align}
\subsection{Isotropic electrostatic and Exchange-correlation energy}
\subsubsection{Second order}
\begin{equation}
    \E{\gamma} = \frac{1}{2}\sum_{A,B}^{N_{atoms}}\sum_{l\in A}\sum_{l'\in B}q_lq_{l'}\gamma_{AB,ll'}
\end{equation}
\subsubsection{Third order}
\begin{equation}
    \E{\Gamma}^{GFN2} = \frac{1}{3}\sum_A^{N_{atoms}}\sum_{l\in A}(q_l)^3\Gamma_{A,l}
\end{equation}
\subsection{Anisotropic electrostatic energy}
\begin{equation}
\begin{split}
    \E{AES} &= \E{q\mu}+\E{q\Theta} + \E{\mu\mu}\\
    &= \frac{1}{2}\sum_{A,B}\{f_3(R_{AB})[q_A(\pmb{\mu}_B^T\pmb{R}_{BA})+q_B(\pmb{\mu}_A^T\pmb{R}_{AB})]\\
    &\quad + f_5(R_{AB})[q_A\pmb{R}_{AB}^T\pmb{\Theta}_B\pmb{R}_{AB}+q_B\pmb{R}_{AB}^T\pmb{\Theta}_A\pmb{R}_{AB}\\
    &\quad -3(\pmb{\mu}_A^T\pmb{R}_{AB})(\pmb{\mu}_B^T\pmb{R}_{AB}) + (\pmb{\mu}_A^T\pmb{\mu}_B)R_{AB}^2] \}\\
\end{split}
\end{equation}
$\pmb{\mu}_A$ is the cumulative atomic dipole moment of atom A and $\pmb{\Theta}_A$ is the corresponding traceless quadrupole moment. Traceless simply means that the sum of the diagonal elements is 0. The curly braces and brackets are used in the same way as normal parenthesis for showing order of operations. $q_A$ is the atomic charge of atom A. 
\begin{align}
    \Theta_A^{\alpha\beta} &= \frac{3}{2} \theta_A^{\alpha\beta} - \frac{\delta_{\alpha\beta}}{2} \left( \theta_A^{xx} + \theta_A^{yy} + \theta_A^{zz} \right)\\
    \theta_A^{\alpha\beta} &= \sum_{l' \in A} \sum_{l} P_{l} \left( \alpha_A D_{ll'}^{\beta} + \beta_A D_{ll'}^{\alpha} - \alpha_A \beta_A S_{ll'} - Q_{ll'}^{\alpha\beta} \right)\\
    q_A &= Z_A - GAP_A\\
    \mu_A^{\alpha} &= \sum_{l' \in A} \sum_{l} P_{l'l} \left( \alpha_A S_{l'l} - D_{l'l}^{\alpha} \right)\\
D_{ll'}^{\alpha} &= \braket{ \phi_{l} | \alpha_i | \phi_{l'}}\\
Q_{ll'}^{\alpha\beta} &= \braket{ \phi_{l} | \alpha_i\beta_i | \phi_{l'}}
\end{align}
$\alpha$ and $\beta$ are Cartesian components labled $(x,y,z)^T$ with atom A being centered in $v_A = (x_i,y_i,y_z)^T$ where i is a form of pointer/label dereferencing. $\delta_{\alpha\beta}$ is just the delta function, i.e is is 1 if $\alpha$ and $\beta$ are the same label and 0 otherwise, this serves to include the term only for the diagonal. 

\begin{equation}
    \pmb{\Theta}_A = 
    \begin{pmatrix}
        \Theta_A^{xx} & \Theta_A^{xy} & \Theta_A^{xz}\\
        \Theta_A^{yx} & \Theta_A^{yy} & \Theta_A^{yz}\\
        \Theta_A^{zx} & \Theta_A^{zy} & \Theta_A^{zz}\\
    \end{pmatrix}
\end{equation}
\begin{equation}
    \pmb{\mu}_A = 
    \begin{pmatrix}
        \mu_A^{x}\\
        \mu_A^{y}\\
        \mu_A^{z}\\
    \end{pmatrix}
\end{equation}
\begin{equation}
    \pmb{R}_{AB} = v_A-v_B
\end{equation}


\begin{align}
    f_n(R_{AB}) &= \frac{f_{damp}(a_n,R_{AB})}{R_{AB}^n}=\frac{1}{R_{AB}^n}\frac{1}{1+6\left(\frac{R_0^{AB}}{R_{AB}}\right)^{a_n}}\\
    R_0^{AB} &= 0.5 ({R^A_0}'+ {R^B_0}')\\
    {R^A_0}' &= \begin{cases}R^A_0 + \frac{R_{max}-R^A_0}{1+exp[-4(CN_A'-N_{val}-\Delta_{val})]} & \text{if }N_{val}\text{ is given}\\5.0 \text{ bohrs} & \text{otherwise}\end{cases}\\
        R_{max} &= 5.0 \text{ bohrs}\\
    \Delta_{val} &= 1.2
\end{align}
$R_0^A$ is a fitted value for 12 elements and 5.0 for the rest. $a_n$ are adjusted global parameters. 



\subsection{Anisotropic XC energy}
\begin{equation}
    \E{AXC} = \sum_A (f^{\mu_A}_{XC}|\pmb{\mu}_A|^2 + f^{\Theta_A}_{XC}||\pmb{\Theta}_A||^2)
\end{equation}
Where $f^{\mu_A}_{XC}$ and $f^{\Theta_A}_{XC}$ are fitted values. What norms are these?

\newpage

\subsection{Dispersion Energy}
\begin{equation}
\begin{split}
  \E{disp}^{D4'} = &-\sum_{A>B} \sum_{n=6,8} s_n \frac{C_n^{AB} (q_A, CN^A_{cov}, q_B, CN^B_{cov})}{R^n_{AB}} f^{(n)}_{damp,BJ} (R_{AB}) \\
  &-s_9 \sum_{A>B>C} \frac{(3cos(\theta_{ABC})cos(\theta_{BCA})cos(\theta_{CAB})+1)C_9^{ABC}(CN_{cov}^A,CN_{cov}^B,CN_{cov}^C)}{(R_{AB} R_{AC} R_{BC})^3} \\
  &\times f^{(9)}_{damp,zero}(R_{AB},R_{AC},R_{BC}).
\end{split}
\end{equation}

\vspace{10pt}
\noindent
The term in the second line is the three-body Axilrod–
Teller–Muto (ATM) (What is this??????) term and the last line is the corresponding zero-damping function for this term.


\vspace{10pt}
\noindent
\(C_6^{AB}\) is the pairwise dipole-dipole dispersion coefficients calculated by numerical integration via the Casimir-Polder relation.
\begin{equation}
  C_6^{AB} = \frac{3}{\pi} \sum_{j} w_j \overline{\alpha}_A (i\omega_j, q_A, CN_{cov}^A)\overline{\alpha}_B (i\omega_j, q_B, CN_{cov}^B)
\end{equation}

\noindent
\(w_j\) are the integration weights, which are derived from a trapeziodal partitioning between the grid points \(j(j \in [1,23])\).

\noindent
The isotropically averaged, dynamic dipole-dipole polarizabilites \(\overline{\alpha}\) at the \(j\)th imaginary frequency \(i\omega_j\) are obtained from the self-consistent D4 model; i.e., they are depending on the covalent coordination number and are also charge dependent.

\begin{equation}
  \overline{\alpha}_A(i\omega_j, q_A, CN_{cov}^A) = \sum_{r}^{N_{A,ref}} \xi_A^r (q_A, q_{A,r}) \overline{\alpha}_{A,r}(i\omega_j, q_{A,r}, CN_{cov}^{A,r}) W_A^r(CN_{cov}^A, CN_{cov}^{A,r})
\end{equation}

\noindent
The Gaussian weighting for each reference system is given by:
\begin{equation}
  W_A^r(CN_{cov}^A, CN_{cov}^{A,r}) = \sum_{j=1}^{N_{gauss}} \frac{1}{\mathcal{N}} \exp\left[-6j \cdot (CN_{cov}^A - CN_{cov}^{A,r})^2\right]
\end{equation}

with
\begin{equation}
  \sum_{r}^{N_{A,ref}} W_A^r(CN_{cov}^A, CN_{cov}^{A,r}) = 1
\end{equation}

\vspace{10pt}
\noindent
\(\mathcal{N}\) is a normalization constant.

\noindent
The number of Gaussian function per reference system \(N_{gauss}\) is mostly one, but equal to three for \(CN_{cov}^{A,r} = 0\) and reference systems with similar coordination number.

\noindent
The carge-dependency is included via the empirical scaling function \(\xi_A^r\).
\begin{equation}
  \xi_A^r(q_A, q_{A,r}) = \exp\left[3\left\{1-\exp\left[4\eta_A\left(1-\frac{Z_A^{eff} + q_{A,r}}{Z_A^{eff} + q_A}\right)\right]\right\}\right]
\end{equation}

\noindent
where \(\eta_A\) is the chemical hardness taken from ref 98.

\noindent
\(Z_A^{eff}\) is the effective nuclear charge of atom A, which has been determined by subtracting the number of core electrons represented by the def2-ECPs in the time-dependent DFT reference calculations.



\vspace{10pt}
\noindent
\(C_8^{AB}\) is calculated recursively from the lowest order \(C_6^{AB}\) coefficients.

\begin{equation}
  C_8^{AB} = 3C_6^{AB} \sqrt{\mathcal{Q}^A\mathcal{Q}^B}
\end{equation}

\begin{equation}
  \mathcal{Q}^A = s_{42} \sqrt{Z^A} \frac{\braket{r^4}^A}{\braket{r^2}^A}
\end{equation}


\vspace{10pt}
\noindent
\(\sqrt{Z^A}\) is the ad hoc nuclear charge dependent factor.

From the original xTB program we can see that \(s_{42}\) is \(0.5\), and \(Z^A\) is the atomic number of A.

\begin{equation}
  \sqrt{0.5 * (\frac{r^4}{r^2} * \sqrt{Z^A})}
\end{equation}

\(\braket{r4}\) and \(\braket{r2}\) are simple multipole-type expectation values derived from atomic densities which are averaged geometrically to get the pair coefficients. (What is 'r', how we get??????) (what is \(s_{42}\)???)

\vspace{20pt}
\noindent
\(CN^A_{cov}\) is the covalent coordination number for atom A.

\vspace{10pt}
\noindent
\(q\) is the atomic charge, so \(q_A\) is the atomic charge for atom A.

\vspace{10pt}
\noindent
The damping and scaling parameters in the
dispersion model are:
\[
  a1 = 0.52 \quad|\quad a2 = 5.0 \quad|\quad s6 = 1.0 \quad|\quad s8 = 2.7 \quad|\quad s9 = 5.0
\]


\vspace{10pt}
\noindent
\(C_9^{ABC}\) is the triple-dipole constant\footnote{\label{dft-d}https://www.researchgate.net/publication/43347348\_A\_Consistent\_and\_Accurate\_Ab\_Initio\_Parametrization\_of\_Density\_Functional\_Dispersion\_Correction\_DFT-D\_for\_the\_94\_Elements\_H-Pu}:
\begin{equation}
  C_9^{ABC} = \frac{3}{\pi} \int_0^\infty \alpha^A(i\omega) \alpha^B(i\omega)\alpha^C(i\omega)d\omega
\end{equation}


\vspace{10pt}
\noindent
The three-body contribution is typically \(<5-10\%\) of \(E_{disp}\), so it is small enough that we can reasonably approximate the coefficients by a geometric mean as\footnoteref{dft-d}:

\begin{equation}
  C_9^{ABC} \approx -\sqrt{C_6^{AB} C_6^{AC} C_6^{BC}}
\end{equation}


\vspace{10pt}
\noindent
\(\theta_{ABC}\) is the angle between the two edges going from B to the other two atoms. \(\theta_{BCA}\) is the angle between the edges going from C to the other two and so on.


\vspace{10pt}
\noindent
BJ = Becke-Johnson

\begin{equation}
  f_n^{damp,BJ}(R_{AB}) = \frac{R_{AB}^n}{R_{AB}^n + (a_1 \cdot R_{AB}^{crit} + a_2)^6}
\end{equation}

\begin{equation}
  R_{AB}^{crit} = \sqrt{\frac{C_8^{AB}}{C_6^{AB}}}
\end{equation}


\begin{equation}
  f_9^{damp,zero}(R_{AB}, R_{AC}, R_{BC}) = \left(1 + 6 \left(\sqrt{\frac{R_{AB}^{crit} R_{BC}^{crit} R_{CA}^{crit}}{R_{AB} R_{BC} R_{CA}}}\right)^{16}\right)^{-1}
\end{equation}
