\chapter{Conclusion}\label{sec:conclusion}

We have shown that massive lockstep-parallelisation on a GPU is a good fit for the problem domain, and for GFN2-xTB in general when considering isomer spaces. Specifically we have explained how the magnitude of parallalism scales with the size and number of isomer spaces, and that this allows us to utilise GPU cores proportional to the total number of fullerenes, which is around $10^9$ in total when considering the isomer spaces \(C_{20}, \cdots, C_{200} \). We have shared ways to evaluate the capability of hardware for our problem domain, in order to estimate the order of parallalism, and to make considerations about the choice of hardware. We have also covered considerations about input structure when generalising lockstep-algorithms for all isomer spaces.

Additionally we have given quantum algorithms for the isotropic terms that scale with the logarithm of the size of the isomer space in terms of qubits. In terms of circuit depth we achieve quadratic scaling in the number of atoms per isomer for each sampling. This is the similar to the classical algorithm but with a much better probability of sampling a good candidate from the isomer space. The probability of sampling an isomer within $\delta$ of the lowest energy after $n$ samples is $1-(1-|\delta|/(|L|-|H|))^n$ in the classical case and $1-(1-(|L|^3-|L-\delta|^3)/(|L|^3-|H|^3))^n$ in the quantum case with $L$ and $H$ being the lowest and highest energies in the isomer space and assuming energies evenly distributed in the interval between.
