\chapter{Conclusion}\label{sec:conclusion}

We have shown that massive lock-step parallelization on a GPU is a good fit for the problem domain, and for GFN2-xTB in particular. We explain necessary considerations to make this approach possible and effective such as restrictions on the input and hardware. Additionally we give quantum algorithms for the anisotropic terms that scale with the logarithm of the size of the isomer space in terms of qubits. In terms of circuit depth we achieve quadratic scaling in the number of atoms per isomer for each sampling. This is the similar to the classical algorithm but with a much better probability of sampling a good candidate from the isomer space. The probability of sampling an isomer within $\delta$ of the lowest energy after $n$ samples is $1-(1-\delta/(H-L))^n$ in the classical case and $1-(1+\delta(2L + \delta)/(L^2-H^2))^n$ in the quantum case with $L$ and $H$ being the lowest and highest energies in the isomer space and assuming energies evenly distributed in the interval between. 
