\chapter{Future Work}\label{sec:future}
In regards to future work, completing the Python prototype is highly relevant. 
An interesting part is implementing the D4' dispersion model. 

Completing the prototype is a crucial step towards the main prize of a fully lockstep-parallel implementation of the algorithm. 

Regarding the quantum algorithms we wanted to investigate using quantum singular value decomposition transforms for some of the terms that are more heavy on tensors. This would be a good thesis topic for anyone interested in quantum algorithm design. It would also be interesting to investigate if any of the approximations in GFN2-xTB are unnecessary in the domain of quantum computing.

We feel that the advent of g-xTB warrants taking another look at the choice of method and potentially adapting our work to the new method. 

Slightly outside of the project scope, we would like to extend the prototype from just working with the energy terms mentioned to include also polarization and excitation energies. Implementing at least one of the solvation models available in the xtb program would also be a welcome addition. The current prototype handles everything as one 'cell', perfect for simulating individual molecules, however extending it to handle repeating cells, such as in crystals, should be a rather small change. 
