\chapter{Theory}
\section{GFN2-xTB}
GFN2-xTB is a part of the GFNn-xTB (Geometry, Frequency, Non covalent, eXtended Tight Binding) family of semi-empirical methods for computational chemistry. 

The method gives good approximations for molecular geometries, vibrational frequencies, and non-covalent interaction energies and also does well on a variety other properties. 
Overall it strives to hit a balance between being accurate, close to the physics, general over a wide range of elements and not too computationally expensive.

This is achieved by approximating a true quantum mechanical simulation, using carefully chosen approximations and parameters. 
In contrast to forcefield methods that often operate on the level of atoms or even functional groups interacting, GFN2-xTB still treats the calculations at the level of individual orbitals in many places. 

GFN2-xTB uses a Self-Consistent Charges (SCC) approach i.e. it makes an initial guess at a density matrix and an energy which it then iteratively refines until both have converged. We are mostly interested in the final energy. 

\subsection{Energy terms}
The terms in GFN2-xTB that determine the energy are as follows
\begin{equation}
E =\E{rep}+\E{disp}^{D4'}+\E{EHT}+\E{\gamma}+\E{AES}+\E{AXC}+\E{\Gamma} + G_{Fermi}
\end{equation}

TODO: snak med Kurt om anisotropiske termer og fermi.

We then have to also factor in the SCC. 
Let us consider the following code sketch in figure \ref{lst:total_energy} which we will fill out in this chapter. The chapter will go over each term and add code snippets illustrating the mathematics in a more concrete way. 
\begin{figure}[H]
\begin{minted}{python}
def get_GFN2_energy(atoms: list[int], positions : list[list[float]]) -> float:
    density = density_initial_guess(atoms)
    overlap, dipol_dipol, charge_quadrupol = overlap(atoms, positions)
    initial_hamiltonian = H0(atoms, positions)
    huckel_theory_matrix = H_EHT(atoms, positions, overlap, initial_hamiltonian)
    charges = mulliken_population_analysis(density,atoms)
    eigen_values = diagonalize(initial_hamiltonian, overlap) # HC=SCe 
    E_repulsion = repulsion(atoms, positions)
    E_dispersion = D4Prime(charges, atoms, positions)
    E_huckel = Huckel(density, extended_huckel_theory_matrix)
    E_anisotropic = AES(charges, overlap, dipol_dipol, charge_quadrupol, positions)
    E_isotropic = IES(charges, positions)
    E_fermi = Fermi(...)
    E = E_repulsion + E_dispersion + E_huckel + E_anisotropic + E_isotropic + E_Fermi
    while not (energy_converged and densities_converged):
        # change densities along the gradients and 
        # update everything using the new density
        # ...
        energy_converged = (E-E_new)**2 < tolerance
        densities_converged = error_squared(density, new_density) < tolerance
        E = E_new
        density = new_density
\end{minted}
\caption{Python code illustrating the main loop of a GFN2-xTB implementation.
Line 2-16 computes the energy in a non SCC manner, line 15-23 iteratively improves the energy using SCC.
Another way to describe this is that the zeroth iteration of SCC happens outside the loop as there is some setup required. 
}
    \label{lst:total_energy}
\end{figure}
Let us define a small helper function to decrease the amount of indentation in the later code examples. This will allow us to easily loop over the orbitals and know which atom and shell each belongs to. 
\begin{figure}[H]
\begin{minted}{python}
def get_orbitals(atoms: list[int]) -> list[tuple[int]]:
    orbitals = []
    for atom_idx,atom in enumerate(atoms):
        for subshell in range(number_of_subshells[atom]):
            l = angular_momentum_of_subshell[atom][subshell] 
            for orbital in range(l*2+1):
                orbitals.append((atom_idx,atom,subshell,orbital))
    return orbitals
\end{minted}
\caption{Python code for generating a convenient list of orbitals to iterate though. }
\end{figure}

And some functions used when we want to signal to the reader that the python list we just created will stay a fixed size. Additionally we get to contain the python list comprehension syntax here. 

\begin{figure}[H]
\begin{minted}{python}
def vector(n: int) -> list[float]:
    return [0.0 for _ in range(n)]

def square_matrix(n: int) -> list[list[float]]:
    return [[0.0 for _ in range(n)] for _ in range(n)]

def square_matrix_of_vectors(n: int, v:int) -> list[list[list[float]]]:
    return [[[0.0 for _ in range(v)] for _ in range(n)] for _ in range(n)]
\end{minted}
    \caption{Python code for creating a $n$ long vector, $n\times n$ matrix and $n\times n \times v$ tensor.}
\end{figure}
\subsection{Constructing the initial density matrix}
Many of the terms use the density matrix as part of their calculation and it is central to the SCC method. In the GFN2 paper the initial density matrix guess is formulated as a superposition of neutral atomic reference densities $P_0 = \sum_{A}P_{A_0}$. In simpler terms this means that we let $P_0$ be a diagonal matrix that is $n$ by $n$ where $n$ is the total number of orbitals across the whole molecule. The values on the diagonal are the fractional number of electrons in the orbitals, the fractional occupations. They are reference densities so we get them in a table we can index into per subshell, thus we just have to divide the electrons evenly between the orbitals in the subshell. The number of orbitals in a subshell is $2l+1$ where l is the quantum number corresponding to the angular momentum of the subshell.
\begin{figure}[H]
\begin{minted}{python}
def density_initial_guess(atoms: list[int]) -> list[list[float]]:
    orbitals = get_orbitals(atoms)
    fractional_occupations = square_matrix(len(orbitals))
    for orbital_idx,(_,atom,subshell,_) in enumerate(orbitals):
        l = angular_momentum_of_subshell[atom][subshell] 
        orbitals_in_subshell = l*2+1 
        electrons_in_subshell = reference_occupations[atom][subshell]
        electrons_per_orbital = electrons_in_subshell/orbitals_in_subshell
        fractional_occupations[orbital_idx][orbital_idx] = electrons_per_orbital
    return fractional_occupations
\end{minted}
\caption{Python code for computation of the initial density matrix $P_0$}
    \label{lst:orbitals}
\end{figure}
For fullerenes the guess is simply all ones on the diagonal as \verb|number_of_subshells[C] = 2|, \verb|angular_momentum_of_subshell[C] = [0,1,0]| and \verb|reference_occupations[C] = [1.0,3.0,0.0]|.
Here \verb|C=5| is the index for carbon. 
Thus \verb|fractional_occupations| will contain a repeating series of $\frac{1}{0\cdot2+1},\frac{3}{1\cdot2+1},\frac{3}{1\cdot2+1},\frac{3}{1\cdot2+1}$ on the diagonal.

\subsection{Constructing the overlap, dipol and quadrupol tensors}
We need the overlap matrix, $S$, for several of the terms.
The related dipol and quadrupol tensors denoted $D$ and $Q$ are also needed and due to their very similar construction we will go over the implementation and mathematics in one fell swoop. 
An element in each tensor is computed in the following way.
\begin{align}
    S_{\nu\mu} &= \braket{\psi_\nu|\psi_\mu} &\forall \nu \in l \in A,\quad\forall\mu \in l' \in B, \\
    D_{\nu\mu}^{u} &= \braket{\psi_\nu|(r_u-{R_P}_u)|\psi_\mu}  &\forall  u\in x,y,z\\
    Q_{\nu\mu}^{uv} &= \braket{\psi_\nu|(r_u-{R_P}_u)(r_v-{R_P}_v)|\psi_\mu} &\forall  v\in x,y,z
\end{align}
These are integrals between two Slater type orbitals (STOs) over the positions $r$. When we write $l \in A$ we mean to iterate though the subshells in every element A and take their angular momentum. $\nu\in l$ means that we iterate though the $2l+1$ orbitals in the subshell with quantum number $m\in\{-l,..,0,..,l\}$. See figure \ref{lst:orbitals} for a code example. ${R_P}$ is the $x,y,z$ position of the product of the orbitals, we will define this later.

For $D$ there are 3 unique values, corresponding to the 3 dimensions, to compute for each pair of orbitals which we will pack into an array. 
Since $(r_u-{R_P}_u)(r_v-{R_P}_v)=(r_v-{R_P}_v)(r_u-{R_P}_u)$ there are 6, rather than 9, unique combinations of dimensions to compute for each element in the $Q$ tensor which we will also pack into an array.
For the sake of clarity we will define the indices corresponding to the combinations of directions, these are however arbitrary and only used for array indexing.  
\begin{figure}[H]
\begin{minted}{python}
    x = 0
    y = 1
    z = 2
    xx = 0
    yy = 1
    zz = 2
    xy = 3
    xz = 4
    yz = 5
\end{minted}
\caption{Constants for the combinations of directions.}
\end{figure}
\begin{figure}[H]
\begin{minted}{python}
    def overlap_dipol_quadrupol(atoms: list[int], positions: list[list[float]])-> tuple[list[list[float]],list[list[list[float]]],list[list[list[float]]]]:
    orbitals = get_orbitals(atoms)
    S = square_matrix(len(orbitals))
    D = square_matrix_of_vectors(len(orbitals),3)
    Q = square_matrix_of_vectors(len(orbitals),6)
    for idx_A, (atom_idx_A,atom_A,subshell_A,orbital_A) in enumerate(orbitals):
        for idx_B, (atom_idx_B,atom_B,subshell_B,orbital_B) in enumerate(orbitals):
            R_A = positions[atom_idx_A]
            R_B = positions[atom_idx_B]
            l_A = angular_momentum_of_subshell[atom_A][subshell_A]
            l_B = angular_momentum_of_subshell[atom_B][subshell_B]
            s,d,q = compute_STO_integrals(...)
            S[idx_A][idx_B] = s
            for dir in [x,y,z]:
                D[idx_A][idx_B][dir] = d[dir]
            for dir in [xx,yy,zz,xy,xz,yz]:
                Q[idx_A][idx_B][dir] = q[dir]
    return S,D,Q
\end{minted}
\caption{Python snippet illustrating construction of the overlap, dipol and quadrupol tensors; $S,D$ and $Q$.}
\end{figure}

To compute the integrals we need to remember that in GFN2-xTB an STO is approximated as a linear combination of gaussian type orbitals:
\begin{equation}
    \begin{split}
        \ket{\psi_\nu} =&\psi(\zeta_{A,l},r-R_A) \\
        = &N_{STO,l}|r-R_A|^{n-1}e^{-\zeta_{A,l}|r-R_A|}Y_l^m(r-R_A)\\
        \approx &\sum_i^{N_{A,l}}c_{i,\nu}N_{GTO,l}(r_x-{R_A}_x)^{l_x}(r_y-{R_A}_y)^{l_y}(r_z-{R_A}_z)^{l_z}e^{-\alpha_{i,\nu}|r-R_A|^2}\\
        =& \sum_i^{N_{A,l}}c_{i,\nu}\phi(\alpha_{i,\nu},r-R_A)\\
        =& \sum_i^{N_{A,l}}c_{i,\nu}\ket{\phi_i}
    \end{split}
\end{equation}
Where $N_{A,l}$ is an element and shell dependant constant, it is the number of GTOs used to approximate the STO. $\zeta$ and $\alpha$ are the slater and gaussian exponents. The contraction coefficient $c$ is a fitted value, it is fitted with the assumption that $\zeta=1$ and to get the value we will actually be using, we need to scale the fitted value by $\zeta^2$. $R_A$ is the position of atom $A$. $N_{STO,l}$ and $N_{GTO,l}$ are normalisation terms. 
We will refer to the terms $(r_u-{R_A}_u)^{l_u}\quad\forall{u\in x,y,z}$ as polynomial prefactors. 
The approximation in terms of GTOs means we can express our overlap in terms of GTO integrals instead:
\begin{align}
    S_{\nu\mu} &=\sum_i^{N_{A,l}}\sum_j^{N_{B,l'}}c_{i,\nu}c_{j,\mu} \braket{\phi_i|\phi_j}\\
    D_{\nu\mu}^u &=\sum_i^{N_{A,l}}\sum_j^{N_{B,l'}}c_{i,\nu}c_{j,\mu} \braket{\phi_i|(r_u-{R_P}_u)|\phi_j}\\
    Q_{\nu\mu}^{uv} &=\sum_i^{N_{A,l}}\sum_j^{N_{B,l'}}c_{i,\nu}c_{j,\mu} \braket{\phi_i|(r_u-{R_P}_u)(r_v-{R_P}_v)|\phi_j}
\end{align}
\begin{figure}[H]
\begin{minted}{python}
def compute_STO_integrals(...) -> tuple[float,list[float],list[float]]:
    number_of_gaussians_A = number_of_gaussians[atom_A][subshell_A]
    number_of_gaussians_B = number_of_gaussians[atom_B][subshell_B]
    slater_exponent_A = slater_exponents[atom_A][subshell_A]
    slater_exponent_B = slater_exponents[atom_B][subshell_B]
    overlap = 0
    dipol = vector(3)
    quadrupol = vector(6)
    for gaussian_i in range(number_of_gaussians_A):
        for gaussian_j in range(number_of_gaussians_B):
            exponent_i = normalised_gaussian_exponent(...)
            exponent_j = normalised_gaussian_exponent(...)
            contraction_i = normalised_contraction_coeficient(...)
            contraction_j = normalised_contraction_coeficient(...)
            s,d,q = compute_GTO_integrals(...)
            overlap += contraction_i*contraction_j*s
            for dir in [x,y,z]:
                dipol[dir] += contraction_i*contraction_j*d[dir]
            for dir in [xx,yy,zz,xy,xz,yz]:
                quadrupol[dir] += contraction_i*contraction_j*q[dir]
    return overlap,dipol,quadrupol
\end{minted}
    \caption{Python snippet illustrating construction of the STO integrals $\braket{\psi_\nu|\cdots|\psi_\mu}$.}
\end{figure}

For an s-orbital the normalisation terms would be:
\begin{align}
    N_{STO,0}&=\sqrt{\frac{\zeta_{A,l}^3}{\pi}}\\
    N_{GTO,0}&=\left(\frac{2\alpha_{i,\nu}}{\pi}\right)^{3/4}
\end{align}
TODO: snak med kurt om den generelle form.

We only need the GTO ones and can compute them in the following way:
\begin{figure}[H]
\begin{minted}{python}
def normalised_gaussian_exponent(atom:int, subshell:int, gaussian:int, slater_exponent:float) -> float:
    normalisation_factor = slater_exponent**2
    return gaussian_exponents[atom][subshell][gaussian]*normalisation_factor

def normalised_contraction_coefficient(atom:int, subshell:int, gaussian:int, gaussian_exponent:float) -> float:
    l = angular_momentum_of_subshell[atom][subshell]
    normalization_factor = (((2.0*gaussian_exponent)/pi)**(3/4)) * (sqrt(4*gaussian_exponent)**l) / sqrt(double_factorial(l))
    return contraction_coeficients[atom][shell][gaussian]*normalisation_factor

def double_factorial(n:int) -> int:
    if n <= 1:
        return 1
    return n*double_factorial(n-2)
\end{minted}
    \caption{Python snippet illustrating normalisation of the gaussian exponents and contraction coefficients.}
\end{figure}

As the GTOs in the integral are real valued functions we can drop the conjugation when we write them out.

For the overlap matrix:
\begin{align}
        \braket{\phi_i|\phi_j} = &\int \phi^*(\alpha_{i,\nu},r-R_A)\phi(\alpha_{j,\mu},r-R_B)\,dr\\
        =&\int \phi(\alpha_{i,\nu},r-R_A)\phi(\alpha_{j,\mu},r-R_B)\,dr\notag
\end{align}

For the dipol tensor:
\begin{align}
        \braket{\phi_i|(r_u-{R_P}_u)|\phi_j} = &\int \phi^*(\alpha_{i,\nu},r-R_A)(r_u-{R_P}_u)\phi(\alpha_{j,\mu},r-R_B)\,dr\\
        =&\int (r_u-{R_P}_u)\phi(\alpha_{i,\nu},r-R_A)\phi(\alpha_{j,\mu},r-R_B)\,dr\notag
\end{align}

For the quadrupol tensor:
\begin{align}
        \braket{\phi_i|(r_u-{R_P}_u)(r_v-{R_P}_v)|\phi_j} = &\int \phi^*(\alpha_{i,\nu},r-R_A)(r_u-{R_P}_u)(r_v-{R_P}_v)\phi(\alpha_{j,\mu},r-R_B)\,dr\\
        =&\int (r_u-{R_P}_u)(r_v-{R_P}_v)\phi(\alpha_{i,\nu},r-R_A)\phi(\alpha_{j,\mu},r-R_B)\,dr\notag
\end{align}
Now we see that we have to compute the products of GTOs. The product of two Gaussians is a new Gaussian centred at a point between the two, $R_P$, which we will finally introduce properly.
\begin{gather}
    \alpha = \alpha_{i,\nu}+\alpha_{j,\mu}\\
    K_{AB} = \left(\frac{2\alpha_{i,\nu}\alpha_{j,\mu}}{\alpha\pi}\right)^\frac{3}{4}e^{-\frac{\alpha_{i,\nu}\alpha_{j,\mu}}{\alpha}|R_A-R_B|^2}\\
    R_P = \frac{\alpha_{i,\nu}R_A+\alpha_{j,\mu}R_B}{\alpha}\\
    \phi(\alpha_{i,\mu},r-R_A)\phi(\alpha_{j,\nu},r-R_B) = K_{AB}\phi(\alpha,r-R_P)
\end{gather}
This rewrite only works if we also shift the polynomial prefactors to be relative to the product centre $R_P$, for all $u\in x,y,z$:
\begin{equation}
\begin{split}
    (r_u-{R_A}_u)^{{l_A}_u} = &\sum_{m_i=0}^{{l_A}_u}\binom{{l_A}_u}{m_i}({R_P}_u-{R_A}_u)^{({l_A}_u-m_i)}(r_u-{R_P}_u)^{{l_A}_u}\\
    =&\sum_{m_i=0}^{{l_A}_u}v_{m_i}(r_u-{R_P}_u)^{{l_A}_u}
\end{split}
\end{equation}
\begin{figure}[H]
\begin{minted}{python}
def shift_polynomial(l_dim, difference_to_P_dim) -> list[float]:
    poly_coefficients = vector(l_dim+1)
    for m in range(l_dim+1):
        poly_coefficients[m] = comb(l_dim, m)*difference_to_P_dim**(l_dim-m)
    return poly_coefficients
\end{minted}
    \caption{Python code illustrating computation of the shifted polynomial coefficients.}
\end{figure}
If we multiply these shifted prefactors we get the prefactors for the product. 
\begin{equation}
\begin{split}
    (r_u-{R_A}_u)^{{l_A}_u}(r_u-{R_B}_u)^{{l_B}_u} = &\sum_{m_i=0}^{{l_A}_u}\sum_{m_j=0}^{{l_B}_u}v_{m_i}v_{m_j}(r_u-{R_P}_u)^{{l_A}_u+{l_B}_u}\\
    =&\sum_t^{{l_A}_u+{l_B}_u}v_t(r_u-{R_P}_u)^t
\end{split}
\end{equation}

In the code we compute the values of $v_t$ as a convolution:
\begin{figure}[H]
\begin{minted}{python}
def convolute(coef_A, coef_B) -> float:
    max_t = len(coef_A)+len(coef_B)
    poly_coefficients = vector(max_t+1)
    for i,ci in enumerate(coef_A):
        for j,cj in enumerate(coef_B):
            poly_coefficients[i+j] += ci*cj
    return poly_coefficients
\end{minted}
    \caption{Python code illustrating computation of the $R_P$ polynomial coefficients via a convolution.}
\end{figure}


To compute the integrals over the GTOs we can split them in terms of the 3 dimensions we are integrating over, $x,y$ and $z$. We can also pull the GTO normalisation terms out front. 

For the overlap matrix:
\begin{align}
        \int \phi(\alpha_{i,\nu},r-R_A)\phi(\alpha_{j,\mu},r-R_B)\,dr=K_{AB}&\int \phi(\alpha,r-R_P)\,dr\\
        =N_{GTO,l_A}N_{GTO,l_B}K_{AB}&\int \phi(\alpha,r_x-{R_P}_x)\,dx\notag\\
            \times&\int \phi(\alpha,r_y-{R_P}_y)\,dy\notag\\
            \times&\int \phi(\alpha,r_z-{R_P}_z)\,dz\notag
\end{align}

For the dipol tensor:
\begin{align}
        \int (r_u-{R_P}_u)\phi(\alpha_{i,\nu},r-R_A)\phi(\alpha_{j,\mu},r-R_B)\,dr=K_{AB}&\int (r_u-{R_P}_u)\phi(\alpha,r-R_P)\,dr\\
        =N_{GTO,l_A}N_{GTO,l_B}K_{AB}&\int (r_u-{R_P}_u)\phi(\alpha,r_x-{R_P}_x)\,dx\notag\\
            \times&\int (r_u-{R_P}_u)\phi(\alpha,r_y-{R_P}_y)\,dy\notag\\
            \times&\int (r_u-{R_P}_u)\phi(\alpha,r_z-{R_P}_z)\,dz\notag
\end{align}

For the quadrupol tensor:
\begin{align}
        \int (r_u-{R_P}_u)(r_v-{R_P}_v)\phi(\alpha_{i,\nu},r-R_A)\phi(\alpha_{j,\mu},r-R_B)\,dr=K_{AB}&\int (r_u-{R_P}_u)(r_v-{R_P}_v)\phi(\alpha,r-R_P)\,dr\\
        =N_{GTO,l_A}N_{GTO,l_B}K_{AB}&\int (r_u-{R_P}_u)(r_v-{R_P}_v)\phi(\alpha,r_x-{R_P}_x)\,dx\notag\\
            \times&\int (r_u-{R_P}_u)(r_v-{R_P}_v)\phi(\alpha,r_y-{R_P}_y)\,dy\notag\\
            \times&\int (r_u-{R_P}_u)(r_v-{R_P}_v)\phi(\alpha,r_z-{R_P}_z)\,dz\notag
\end{align}
Each of these integrals can now be expanded using the analytical solution to Gaussian integrals of the form $\int x^t e^{-ax^2}\,dx$\cite{}. Let us briefly focus on the integrals over $x$ as they are all identical.

TODO: find den rigtig reference for eq 34 https://mathworld.wolfram.com/GaussianIntegral.html

For the overlap matrix:
\begin{align}
    \int \phi(\alpha,r_x-{R_P}_x)\,dx = \int \sum_{t=0}^{{l_A}_x+{l_B}_x}v_t(r_x-{R_P}_x)^te^{-\alpha|r_x-{R_P}_x|^2}\,dx\\
        =  \sum_{t=0}^{{l_A}_x+{l_B}_x}v_t\int(r_x-{R_P}_x)^{t}e^{-\alpha|r_x-{R_P}_x|^2}\,dx\notag\\
        =  \sum_{\{t\in0,..,{l_A}_x+{l_B}_x | odd(t)\}}v_{t}\frac{\left(\frac{t-1}{2}\right)!}{2a^{\frac{t+1}{2}}}+
        \sum_{\{t\in0,..,{l_A}_x+{l_B}_x | even(t)\}}v_{t}\frac{\left(t-1\right)!!}{2^{\frac{t}{2}+1}a^{\frac{t+1}{2}}}\sqrt{\frac{\pi}{\alpha}}\notag
\end{align}

For the dipol tensor:
\begin{align}
    \int (r_u-{R_P}_u)\phi(\alpha,r_x-{R_P}_x)\,dx = \int \sum_{t=0}^{{l_A}_x+{l_B}_x}v_t(r_u-{R_P}_u)(r_x-{R_P}_x)^te^{-\alpha|r_x-{R_P}_x|^2}\,dx\\
        =  \sum_{t=0}^{{l_A}_x+{l_B}_x}v_t\int(r_x-{R_P}_x)^{t+\delta_{ux}}e^{-\alpha|r_x-{R_P}_x|^2}\,dx\notag\\
        =  \sum_{t=\delta_{ux}}^{{l_A}_x+{l_B}_x+\delta_{ux}}v_{t-\delta_{ux}}\int(r_x-{R_P}_x)^{t}e^{-\alpha|r_x-{R_P}_x|^2}\,dx\notag\\
        =  \sum_{\{t\in\delta_{ux},..,{l_A}_x+{l_B}_x+\delta_{ux} | odd(t)\}}v_{t-\delta_{ux}}\frac{\left(\frac{t-1}{2}\right)!}{2a^{\frac{t+1}{2}}}+
        \sum_{\{t\in\delta_{ux},..,{l_A}_x+{l_B}_x+\delta_{ux} | even(t)\}}v_{t-\delta_{ux}}\frac{\left(t-1\right)!!}{2^{\frac{t}{2}+1}a^{\frac{t+1}{2}}}\sqrt{\frac{\pi}{\alpha}}\notag
\end{align}

For the quadrupol tensor:
\begin{align}
    \int (r_u-{R_P}_u)(r_v-{R_P}_v)\phi(\alpha,r_x-{R_P}_x)\,dx \\
    = \int \sum_{t=0}^{{l_A}_x+{l_B}_x}v_t(r_u-{R_P}_u)(r_v-{R_P}_v)(r_x-{R_P}_x)^te^{-\alpha|r_x-{R_P}_x|^2}\,dx\notag\\
        =  \sum_{t=0}^{{l_A}_x+{l_B}_x}v_t\int(r_x-{R_P}_x)^{t+\delta_{ux}+\delta_{vx}}e^{-\alpha|r_x-{R_P}_x|^2}\,dx\notag\\
        =  \sum_{t=\delta_{ux}+\delta_{vx}}^{{l_A}_x+{l_B}_x+\delta_{ux}}v_{t-\delta_{ux}-\delta_{-vx}}\int(r_x-{R_P}_x)^{t}e^{-\alpha|r_x-{R_P}_x|^2}\,dx\notag\\
        =  \sum_{\{t\in\delta_{ux}+\delta_{vx},..,{l_A}_x+{l_B}_x+\delta_{ux}+\delta_{vx} | odd(t)\}}v_{t-\delta_{ux}-\delta_{vx}}\frac{\left(\frac{t-1}{2}\right)!}{2a^{\frac{t+1}{2}}}+\notag\\
        \sum_{\{t\in\delta_{ux}+\delta_{vx},..,{l_A}_x+{l_B}_x+\delta_{ux}+\delta_{vx} | even(t)\}}v_{t-\delta_{ux}-\delta_{vx}}\frac{\left(t-1\right)!!}{2^{\frac{t}{2}+1}a^{\frac{t+1}{2}}}\sqrt{\frac{\pi}{\alpha}}\notag
\end{align}

TODO: Er det her rigtigt og i så fald hvor kommer $R_P^{(0,1,2)}$ termet i deres, og nu vores, kode fra?

The final sums are something we can compute. For the most part the analytical solution to the integrals is independent of the direction so we can precompute the independent part: 
\begin{figure}[H]
\begin{minted}{python}
def compute_gaussian_integral_factors(alpha:float, l_A:int, l_B:int)-> list[float]:
    factors = vector(l_A+l_B+3)
    for t in range(l_A+l_B+3):
        if t % 2 == 0:
            factors[t] = factorial((t-1)//2)/(2*alpha*((t+1)//2))
        else:
            factors[t] = (double_factorial(t-1)/(2**(t/2+1)*alpha**((t+1)//2))) * sqrt(pi/alpha)
    return factors
\end{minted}
    \caption{Python code illustrating computation of the integral factors for even and odd powers}
\end{figure}
We compute an extra $2$ factors to accommodate $\delta_{ux}+\delta_{vx}$

\begin{figure}[H]
\begin{minted}{python}
def compute_GTO_integrals(...)-> tuple[float,list[float],list[float]]:
    alpha = exponent_i + exponent_j
    exponents = exponent_i * exponent_j
    K_AB = ((2*exponents)/(alpha*pi))**(3/4) * e**(-(exponents/alpha)*distance)
    integral_factors = compute_gaussian_integral_factors(alpha, l_A, l_B)
    zeroth_moment = vector(3)
    first_moment = vector(3)
    second_moment = vector(3)
    for dim in [x,y,z]:
        gaussian_product_center_in_dim = (exponent_i*R_A[dim]+exponent_j*R_B[dim])/alpha
        centre_relative_to_A = gaussian_product_center_in_dim-R_A[dim]
        centre_relative_to_B = gaussian_product_center_in_dim-R_B[dim]
        l_A_dim = angular_momentum_in_dimension[l_A][orbital_A][dim]
        l_B_dim = angular_momentum_in_dimension[l_B][orbital_B][dim]
        l_max_dim = max(l_A_dim, l_B_dim)
        vmis = shift_polynomial(l_A_dim, centre_relative_to_A)
        vmjs = shift_polynomial(l_B_dim, centre_relative_to_B)
        vts = convolute(vmis, vmjs)
        for t, vt in enumerate(vts):
            zeroth_moment[dim] += vt * integral_factors[t+m]
            for m in range(1+1):
                first_moment[dim] += comb(1,m) * gaussian_product_center_in_dim**(1-m) * vt * integral_factors[t+m]
            for m in range(2+1):
                second_moment[dim] += comb(2,m) * gaussian_product_center_in_dim**(2-m) * vt * integral_factors[t+m]

    overlap = zeroth_moment[x]*zeroth_moment[y]*zeroth_moment[z]

    dipol = vector(3)
    dipol[x] = first_moment[x]*zeroth_moment[y]*zeroth_moment[z]
    dipol[y] = zeroth_moment[x]*first_moment[y]*zeroth_moment[z]
    dipol[z] = zeroth_moment[x]*zeroth_moment[y]*first_moment[z]

    quadrupol = vector(3)
    quadrupol[xx] = second_moment[x]*zeroth_moment[y]*zeroth_moment[z]
    quadrupol[yy] = zeroth_moment[x]*second_moment[y]*zeroth_moment[z]
    quadrupol[zz] = zeroth_moment[x]*zeroth_moment[y]*second_moment[z]
    quadrupol[xy] = first_moment[x]*first_moment[y]*zeroth_moment[z]
    quadrupol[xz] = first_moment[x]*zeroth_moment[y]*first_moment[z]
    quadrupol[yz] = zeroth_moment[x]*first_moment[y]*first_moment[z]
    return overlap, dipol, quadrupol
\end{minted}
    \caption{Python code illustrating computation of the integral $\braket{\phi_i|\cdots|\phi_j}$ using the dimension-wise decomposition.}
\end{figure}
%\begin{equation}
%N^{G}_{i}(\alpha_i,\ell_x,\ell_y,\ell_z)
%= \sqrt{\left(\frac{2\alpha_i}{\pi}\right)^{3/2}
%\frac{(4\alpha_i)^{\ell}}{(2\ell_x-1)!!\,(2\ell_y-1)!!\,(2\ell_z-1)!!}},
%\end{equation}

\subsection{Building the Hückel matrix}
As a part of computing the Hückel energy we need to construct the Hückel matrix.
The Hückel matrix comes from extended Hückel theory. Before we can construct it we need a few definitions. 
First the GFN2 type coordination number for atom $A$ is
\begin{gather}
    \begin{split}
        CN_A' = &\sum_{B\neq A}\left(1+e^{-10\left(\frac{4\left(R_{A,cov}+R_{B,cov}\right)}{3R_{AB}}-1\right)}\right)^{-1}\left(1+e^{-20\left(\frac{4\left(R_{A,cov}+R_{B,cov}+2\right)}{3R_{AB}}-1\right)}\right)^{-1}
    \end{split}
\end{gather}
Where $R_{A,cov}$ is the covalent atomic radius of $A$ specified for the GFN2 method. $R_{AB}$ is the euclidean distance between $A$ and $B$.
Now with $EN_A$ being the Pauling electronegativity and $k_EN=0.02$ as well as $k_{A,l}^{\text{poly}}$ being GFN2 specific constants we define the following 3 functions:
\begin{gather}
    X(EN_A,EN_B) = 1 + k_{EN}(EN_A-EN_B)^2\\
    \Pi(R_{AB},l,l') = \left(1 + k^{\text{poly}}_{A,l}\left(\frac{R_{AB}}{R_{A,cov}+R_{B,cov}}\right)^\frac{1}{2}\right)\left(1 + k^{\text{poly}}_{B,l'}\left(\frac{R_{AB}}{R_{A,cov}+R_{B,cov}}\right)^\frac{1}{2}\right)\\
    Y(\zeta_{A,l},\zeta_{B,l'}) = \left(\frac{2\sqrt{\zeta_{A,l}\zeta_{B,l'}}}{\zeta_{A,l}+\zeta_{B,l'}}\right)^\frac{1}{2}
\end{gather}

The Hückel matrix is then calculated as follows:
\begin{align}
    H_{\nu\nu}^{EHT} &= H_A^l - H_{CN_A}CN_A'\\
    \mn{H^{EHT}} &= \frac{1}{2}K^{ll'}_{AB}\mn{S}(H^{EHT}_{\mu\mu}+H^{EHT}_{\nu\nu})\\
    &\cdot X(EN_A,EN_B)\notag\\
    &\cdot \Pi(R_{AB},l,l')\notag\\
    &\cdot Y(\zeta^A_l,\zeta^B_{l'}), \quad\quad \forall \mu \in l \in A, \nu \in l' \in B\notag
\end{align}

$K^{ll'}_{AB}$ is an element and shell specific fitted constant however, in GFN2 it only depends on the shells. 
$S_{\mu\nu}=\braket{\phi_\mu|\phi_\nu}$ is the just introduced overlap of the orbitals. 
$H^l_A$ and $H^l_{CN'_A}$ are both fitted constants. $EN_A$ is the electronegativity of the element of atom A, given in the original \texttt{xtb} code. 

\begin{figure}[H]
\begin{minted}{python}
def huckel_matrix(atoms: list[int], positions: list[list[float]], overlap: list[list[float]]) -> list[list[float]]:
    orbitals = get_orbitals(atoms)
    H_EHT = square_matrix(len(orbitals))
    CN = get_coordination_numbers(atoms,positions)
    for orbital_idx, (atom_idx,atom,subshell,orbital) in enumerate(orbitals):
        CN_A = CN[atom_idx]
        H_A = self_energy[atom][subshell] # constant
        H_CN_A = GFN2_H_CN_A[atom][subshell] # constant
        H_EHT[orbital_idx][orbital_idx] = H_A - H_CN_A*CN_A

    for idx_A,(atom_A_idx,atom_A,subshell_A,_) in enumerate(orbitals):
        l_A = angular_momentum_of_subshell[atom_A][subshell_A]
        EN_A = electro_negativity[atom_A]
        R_A = positions[atom_A_idx]
        Rcov_A = covalent_radii[atom_A]
        k_poly_A = k_poly[atom_A][l_A]
        for idx_B,(atom_B_idx,atom_B,subshell_B,_) in enumerate(orbitals):
            if idx_A == idx_B:
                continue
            l_B = angular_momentum_of_subshell[atom_B][subshell_B]
            EN_B = electro_negativity[atom_B]
            R_B = positions[atom_B_idx]
            Rcov_B = covalent_radii[atom_B]
            k_poly_B = k_poly[atom_B][l_B]
            K_ll = GFN2_K_AB[l_A][l_B]
            delta_EN_squared = (EN_A-EN_B)**2
            k_EN = 0.02
            X = 1+k_EN*delta_EN_squared
            R_AB = euclidean_distance(R_A,R_B) 
            Rcov_AB = Rcov_A + Rcov_B 
            PI = (1+k_poly_A*sqrt(R_AB/Rcov_AB)) * (1+k_poly_B*sqrt(R_AB/Rcov_AB))
            slater_exp_A = slater_exponent[atom_A][l_A]
            slater_exp_B = slater_exponent[atom_B][l_B]
            Y = sqrt((2*sqrt(slater_exp_A*slater_exp_B)) / (slater_exp_A+slater_exp_B))
            H_nn = H_EHT[idx_A][idx_A]
            H_mm = H_EHT[idx_B][idx_B]
            S_nm = overlap[idx_B][idx_B]
            H_EHT[idx_A][idx_B] = k_ll*(1/2)*(H_nn+H_mm)*S_nm*Y*X*PI
    return H_EHT
\end{minted}
\caption{Python code illustrating construction of the Hückel matrix}
\end{figure}
Using this we can now compute the Hückel energy where $P$ is the density matrix at this point in the SCC procedure:
\begin{equation}
    \E{EHT} = \sum_{\mu\nu}P_{\mu\nu}H^{EHT}_{\mu\nu}\\
\end{equation}



\begin{figure}[H]
\begin{minted}{python}
def get_GFN2_Huckel_energy(density_matrix:list[list[int]], huckel_matrix:list[list[int]]) -> float:
    E_EHT = 0
    orbitals = len(density_matrix)
    for shell_mu in range(orbitals):
        for shell_nu in range(orbitals):
            E_EHT += density_matrix[shell_nu][shell_mu] * huckel_matrix[shell_mu][shell_nu]
    return E_EHT
\end{minted}
\caption{Python code illustrating computation of the Hückel energy}
\end{figure}


\subsection{Repulsion energy}
The repulsion energy in GFN2-xTB is computed classically using the following formula:
\begin{align}
    \E{rep} &= \frac{1}{2}\sum_{A}\sum_{B\neq A}\frac{Z^{eff}_A Z^{eff}_B}{R_{AB}}e^{-\sqrt{a_Aa_B}(R_{AB})^{(k_f)}}\\
k_f &= \begin{cases}1 & if A,B\in\{\text{H},\text{He}\}\\\frac{3}{2}&otherwise\end{cases} 
\end{align}
The effective nuclear charge $Z^{eff}$ and $a$ are variables fitted for each element. 
$R_{AB}$ is the distance between the A and B atoms.
This can be implemented in the following way:

\begin{figure}[H]
\begin{minted}{python}
H = 0
He = 1
def get_GFN2_repulsion_energy(atoms: list[int], positions : list[list[float]]) -> float:
    sum = 0
    for idx_A, (atom_A, position_A) in enumerate(zip(atoms, positions)):
        for idx_B, (atom_B, position_B) in enumerate(zip(atoms, positions)):
            if idx_A == idx_B:
                continue
            k_f = 3/2
            if atom_A in [H, He] and atom_B in [H, He]:
                k_f = 1
            R_AB = euclidean_distance(position_A, position_B)
            alpha_A = repulsion_alpha[atom_A]
            alpha_B = repulsion_alpha[atom_B]
            Y_A = effective_nuclear_charge[atom_A]
            Y_B = effective_nuclear_charge[atom_B]
            sum += ((Y_A*Y_B)/R_AB) * e ** ( - sqrt(alpha_A*alpha_B) * R_AB**k_f )
    E_rep = (1/2)*sum
    return E_rep
\end{minted}
\caption{Python code illustrating computation of the repulsion energy}
\end{figure}

\subsection{Isotropic Energy}
The isotropic energy terms $\E{\gamma}$ and $\E{\Gamma}$ require us to first do a Mulliken population analysis to get the subshell-wise partial Mulliken charges $q_{A,l}$.
\begin{equation}
    q_{A,l} = \text{TODO:Snak med kurt}\\
\end{equation}

With $\eta_A$ being an element specific constant and $k_{A,l}$ being a shell specific constant we define, both specific to the parametrisation of GFN2:
\begin{equation}
    \eta = \frac{(1+k_{A,l})\eta_A+(1+k_{B,l'})\eta_B}{2}\\
\end{equation}

\begin{equation}
    \gamma_{AB,ll'} = \sqrt{\frac{1}{R_{AB}^2+\eta^{-2}}}
\end{equation}
We can then define one of the two parts of the isotropic energy, $\E{\gamma}$
\begin{equation}
    \E{\gamma} = \frac{1}{2}\sum_{A,B}^{N_{atoms}}\sum_{l\in A}\sum_{l'\in B}q_{A,l}q_{B,l'}\gamma_{AB,ll'}\label{eq:gamma}
\end{equation}
With the element and shell specific GFN2 constant $\Gamma_{A,l}$ we also get the definition for the other part:
\begin{equation}
    \E{\Gamma} = \frac{1}{3}\sum_A^{N_{atoms}}\sum_{l\in A}(q_{A,l})^3\Gamma_{A,l} \label{eq:Gamma}
\end{equation}

