\chapter{Reflections}\label{sec:reflection}
In this section we will reflect on our choices and approaches to highlight what we think worked well. We will also share possible changes worth considering for anyone pursuing to continue this project.%We will also share changes in the landscape that have occured since the start of this project.

We have primarily made use of the Python programming language for the prototype implementation of GFN2-xTB. This has worked well due to Python's simple syntax, runtime checks, and abstractions of low-level details such as memory management. These traits have let us focus on correctness rather than specifics about the code and language.

The package manager and functional language Nix has given us similar advantages by providing us with reproducable development environments, a flexible solution for software packaging, and a way to combine this to make a reproducable testing pipeline. %Because of this, we have been able to focus on troubleshooting the actual implementation rather than system specific problems.

Nix is known to have a rather steep learning curve, which deters many, but given our experiences, we can wholeheartedly recommend this workflow. The testing setup with Nix has been especially valuable, continuing to pay dividends by ensuring correctness as we transition from a prototype to a high performance implementation.

The xTB family of algorithms presented by Grimme et al. is still being actively developed. In fact g-xTB came out a little over half way though this project and is rather impressive regarding accuracy and breadth of application. We urge anyone continuing this project to consider if the resulting changes impact fullerenes enough to warrant switching to the new method. We were unable to make this assessment as the reference implementation has still not landed in the tblite repository, and the associated paper is still only a preprint available on chemRxiv\cite{g-xtb}. 
The eventual advent of g-xTB may justify adapting our work to the new method. 

Completing the Python prototype is highly relevant as it is a crucial step towards the main prize of a fully lock-step parallel implementation of the algorithm. 
An interesting stand-alone part of the method is implementing the D4' dispersion model. 

Slightly outside of the project scope, we would like to extend the prototype from just computing the energy terms mentioned to include also polarization and excitation energies. Implementing at least one of the solvation models available in the xtb program would also be a welcome addition. The current prototype handles everything as one 'cell', perfect for simulating individual molecules, however extending it to handle repeating cells, such as in crystals, should be a rather small change. 

Regarding quantum algorithms we want to investigate using quantum singular value decomposition transforms for some of the terms that are more heavy on tensors. This would be a good thesis topic for anyone interested in quantum algorithm design. It would also be interesting to investigate if any of the approximations in GFN2-xTB are unnecessary in the domain of quantum computing.
