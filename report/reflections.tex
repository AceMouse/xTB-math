\chapter{Reflections}\label{sec:reflection}

In this section we will reflect on our choices and approaches to highlight what we think worked well. We will also share possible changes worth considering for anyone pursuing to continue this project.%We will also share changes in the landscape that have occured since the start of this project.

We have primarily made use of the Python programming language for the prototype implementation of GFN2-xTB. This has worked well due to Python's simple syntax, large ecosystem, runtime checks, and abstractions of low-level details such as memory management. These traits has let us focus on correctness rather than specifics about the code and language.

The package manager and functional language Nix has given us similar advantages by providing us with reproducable development environments, a flexible solution for software packaging, and a way to combine this to make a reproducable testing pipeline. %Because of this, we have been able to focus on troubleshooting the actual implementation rather than system specific problems.

Nix is known to have a rather steep learning curve, which deters many, but given our experiences, we can wholeheartedly recommend this workflow. The testing setup with Nix has been especially valuable, continuing to pay dividends by ensuring correctness as we transition from a prototype to a high performance implementation.

The xTB family of algorithms presented by Grimme et al. is still being actively developed. In fact, the new method g-xTB was introduced during the later half of this project, and it appears to be rather impressive regarding accuracy and breadth of applicability. We urge anyone pursuing to continue this project, to research the impact this new method has on fullerene isomerspaces, and conclude accordingly whether the results warrant switching away from GFN2-xTB. We were not able investigate this, as the g-xTB method is not yet part of the tblite project, and only a preprint of the research paper is currently available.% Reference to chemRxiv? 
