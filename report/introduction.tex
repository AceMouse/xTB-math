\chapter{Introduction}
Calculating the properties of molecules is of major interest to any chemist. Computational approaches have become astonishingly precise and are of great use. A challenge is that the more accurate methods usually take weeks or months to run on large molecules. If we want results in a more timely manner we either have to spend more money on hardware, make faster software and algorithms or find a way to do more targeted calculations. 

In this thesis we will contribute to James E. Avery's efforts to develop an efficient screening pipeline for fullerenes. The concept of a screening pipeline applies when you want to calculate the properties of a large amount of molecules e.g. large isomer spaces in search of the best molecules for your purposes. In such a pipeline progressively slower techniques are used in each step while discarding poor candidate molecules in each. 

For our purpose we work with the isomer spaces $C_n$ where their size is on the order of $O(n^9)$.
These quickly become very large and so the first step in the screening pipeline must be very efficient. This motivates picking a fast approximate algorithm and implementing a highly optimised version.
We landed on GFN2-xTB for this role. It is part of a family of algorithms for computing the geometries, frequencies and noncovalent bonds using extended tight binding methods. It is semi-emperical and already decently fast. 

Calculating the properties of a molecule in $C_200$ takes about 30 seconds on a good desktop using the GFN2-xTB reference implementation. A quick back of the envelope calculation then puts the time to calculate all the molecules in the $C_200$ isomer space at around 200 years. To bring this down we now only have two levers to pull, acquiring more hardware or faster software. 

Thus this thesis will provide a blueprint for accelerating the calculation of entire isomer spaces of highly similar molecules in GFN2-xTB, as well as an in-depth look at the GFN2-xTB algorithm. We will exploit the fact that fullerenes are entirely made up of carbon atoms to provide lock-step parallel algorithms suitable for running on GPUs and taking advantage of their high floating point operation throughput.

We will also evaluate the opportunities for using quantum computing to speed up the search for good candidates. We will design quantum circuits for computing the GFN2-xTB isotropic energy terms and discuss how this can be applied to superposition's of molecules from an isomer space. 

In chapter \ref{sec:code_struct} we will introduce the repository structure, and in chapter \ref{sec:gfn2} we will explain the GFN2-xTB method including code snippets. In chapter \ref{sec:hppc} we will go over the practicalities of designing and implementing a lock-step parallel algorithm on the GPU. In chapter \ref{sec:quant} we will Design and analyse quantum circuits as an alternative approach to calculating the anisotropic energy terms of GFN2-xTB. In chapter \ref{sec:related} we will briefly discuss related projects. Chapter \ref{sec:meth} introduces our implementation approach and how we ensure correctness in relation to the reference implementation. In chapters \ref{sec:results}, \ref{sec:reflection}, \ref{sec:future} and \ref{sec:conclusion} we reflect on and conclude the project while reiterating our results.
