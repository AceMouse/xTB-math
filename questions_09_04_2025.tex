\documentclass{article}
\usepackage{amsmath}
\usepackage{braket}

\makeatletter
\newcommand{\q}[1]{\textbf{QUESTION: #1}}
\newcommand\E{\@ifnextchar[{\@with}{\@without}}
\def\@with[#1]#2{E_{#2}^{(#1)}}
\def\@without#1{E_{#1}}
\makeatother

\makeatletter
\newcommand\footnoteref[1]{\protected@xdef\@thefnmark{\ref{#1}}\@footnotemark}
\makeatother

\begin{document}
\section{Total Energy for GFN2-xTB}
Most equations etc. are from the xTB review paper or the GFN2-xTB paper.
\begin{equation}
\begin{split}
\E{GFN2-xTB} &= \E[0]{rep}+\E[0,1,2]{disp}+\E[1]{EHT}+\E[2]{IES+IXC}+\E[2]{AES+AXC}+\E[3]{IES+IXC}\\
&=\E{rep}+\E{disp}^{D4'}+\E{EHT}+\E{\gamma}+\E{AES}+\E{AXC}+\E{\Gamma}^{GFN2}
\end{split}
\end{equation}
\subsection{Repulsion Energy}
\begin{equation}
\begin{split}
\E{rep} &= \frac{1}{2}\sum_{A,B}\frac{Z^{eff}_A Z^{eff}_B}{R_{AB}}e^{-\sqrt{a_Aa_B}(R_{AB})^{(k_f)}}\\
k_f &= \begin{cases}1 & if A,B\in\{\text{H},\text{He}\}\\\frac{3}{2}&otherwise\end{cases} 
\end{split}
\end{equation}
$Z^{eff}$ and $a$ are variables fitted for each element. A,B are the labels of atoms. 
Since we only have C and H in our systems we can simplify this quite a bit in code. 
$R_{AB}$ is the distance between the A and B atoms.
\subsection{Extended Hückel Theory Energy}
\begin{equation}
\begin{split}
\E{EHT} &= \sum_{\mu\nu}P_{\mu\nu} + H^{EHT}_{\mu\nu}
\end{split}
\end{equation}
where $\mu$ and $\nu$ are AO indecies, $l$ and $l'$ index shells. Both AO's are associated with an atom labled A and B. 
\begin{equation}
\begin{split}
P_{\mu\nu} &= P^{(0)}_{\mu\nu}+ \delta P_{\mu\nu}\\
    P^{(0)}_{\mu\nu}&=??\\ 
    \delta P_{\mu\nu} &=??\\
H^{EHT}_{\mu\nu} &= \frac{1}{2}K^{ll'}_{AB}S_{\mu\nu}(H_{\mu\mu}+H_{\nu\nu})\\&\cdot X(EN_A,EN_B)\\&\cdot \Pi(R_{AB},l,l')\\&\cdot Y(\zeta^A_l,\zeta^B_{l'}), \forall \mu \in l(A), \nu \in l'(B)
\end{split}
\end{equation}

\noindent
$S_{\mu\nu}=\braket{\phi_\mu|\phi_\nu}$ is just the overlap of the orbitals.

\vspace{10pt}
\noindent
\q{How do we calculate the density matrix or the zeroth order and delta terms? }

\vspace{10pt}
\noindent
\q{How do we calculate the orbitals from the Slater exponents?}

\subsection{Isotropic electrostatic and XC energy}
\subsubsection{Second order}
\begin{equation}
\begin{split}
    \E{\gamma} &= \frac{1}{2}\sum_{A,B}^{N_{atoms}}\sum_{l\in A}\sum_{l'\in B}q_lq_{l'}\gamma_{AB,ll'}\\
    \gamma_{AB,ll'} &= \frac{1}{\sqrt{R_{AB}^2+\eta^{-2}_{AB,ll'}}}\\
    \eta_{AB,ll'}&=\frac{1}{2}\left[\eta_A(1+k_A^l)+\eta_B(1+k_B^{l'})\right]
\end{split}
\end{equation}
$q_l$ is a partial muliken charge. $\eta_A$ and $\eta_B$ are element-specific fit parameters, while $k_A^l$ and $k_B^{l'}$ are element-specific scaling factors for the individual shells ($k_A^l=0$ when $l=0$).
\subsubsection{Third order}
\begin{equation}
\begin{split}
    \E{\Gamma}^{GFN2} &= \frac{1}{3}\sum_A^{N_{atoms}}\sum_{l\in A}(q_l)^3K^\Gamma_l\Gamma_A
\end{split}
\end{equation}
$K^\Gamma_l$ is a shell specific constant common for all elements and $\Gamma_A$ is an element specific constant. 

\noindent
\q{We can do all of this, except get the partial charges. We assume we have to do Self Consistent Charge, but how we do that is not clear from the GFN2-xTB paper, the xTB review paper or Frank Jensen CH 7.6. Any hints?}
\subsection{Anisotropic electrostatic energy}
\begin{equation}
\begin{split}
    \E{AES} &= \E{q\mu}+\E{q\Theta} + \E{\mu\mu}\\
    &= \frac{1}{2}\sum_{A,B}\{f_3(R_{AB})[q_A(\pmb{\mu}_B^T\pmb{R}_{BA})+q_B(\pmb{\mu}_A^T\pmb{R}_{AB})]\\
    &\quad + f_5(R_{AB})[q_A\pmb{R}_{AB}^T\pmb{\Theta}_B\pmb{R}_{AB}+q_B\pmb{R}_{AB}^T\pmb{\Theta}_A\pmb{R}_{AB}\\
    &\quad -3(\pmb{\mu}_A^T\pmb{R}_{AB})(\pmb{\mu}_B^T\pmb{R}_{AB}) + (\pmb{\mu}_A^T\pmb{\mu}_B)R_{AB}^2] \}\\
\end{split}
\end{equation}
\q{Is $\pmb{R}_{AB} = (x_i-x_j,y_i-y_j,z_i-z_j)^T$ or maybe $\pmb{R}_{AB} = (x_i+x_j,y_i+y_j,z_i+z_j)^T$, if atoms A and B are centered in $(x_i,y_i,z_i)^T$ and $(x_j,y_j,z_j)^T$? The latter would be weird as then $\pmb{R}_{AB}=\pmb{R}_{BA}$ and why both then?}

\noindent
$\pmb{\mu}_A$ is the cumulative atomic dipole moment of atom A and $\pmb{\Theta}_A$ is the corresponding traceless quadrupole moment. The curly braces and brackets are used in the same way as normal parenthesis for showing order of operations. $q_A$ is the atomic charge of atom A. 
\begin{align}
    \Theta_A^{\alpha\beta} &= \frac{3}{2} \theta_A^{\alpha\beta} - \frac{\delta_{\alpha\beta}}{2} \left( \theta_A^{xx} + \theta_A^{yy} + \theta_A^{zz} \right)\\
    \theta_A^{\alpha\beta} &= \sum_{\kappa \in A} \sum_{\lambda} P_{\lambda} \left( \alpha_A D_{\lambda\kappa}^{\beta} + \beta_A D_{\lambda\kappa}^{\alpha} - \alpha_A \beta_A S_{\lambda\kappa} - Q_{\lambda\kappa}^{\alpha\beta} \right)\\
    q_A &= Z_A - \sum_{\kappa \in A} \sum_{\lambda} P_{\kappa\lambda} S_{\lambda\kappa}\\
    \mu_A^{\alpha} &= \sum_{\kappa \in A} \sum_{\lambda} P_{\kappa\lambda} \left( \alpha_A S_{\kappa\lambda} - D_{\kappa\lambda}^{\alpha} \right)\\
D_{\lambda\kappa}^{\alpha} &= \braket{ \phi_{\lambda} | \alpha_i | \phi_{\kappa}}\\
Q_{\lambda\kappa}^{\alpha\beta} &= \braket{ \phi_{\lambda} | \alpha_i\beta_i | \phi_{\kappa}}
\end{align}
$\alpha$ and $\beta$ are Cartesian components. 

\noindent
\q{Are we correct in the following, assuming that atom A is centered in $(x_i,y_i,z_i)^T$?
\begin{equation*}
    \pmb{\Theta}_A = 
    \begin{pmatrix}
        \Theta_A^{xx} & \Theta_A^{xy} & \Theta_A^{xz}\\
        \Theta_A^{yx} & \Theta_A^{yy} & \Theta_A^{yz}\\
        \Theta_A^{zx} & \Theta_A^{zy} & \Theta_A^{zz}\\
    \end{pmatrix}
\end{equation*}
\begin{equation*}
    \pmb{\mu}_A = 
    \begin{pmatrix}
        \mu_A^{x}\\
        \mu_A^{y}\\
        \mu_A^{z}\\
    \end{pmatrix}
\end{equation*}}

\noindent
\q{Is $\delta$ 1 if the labels match and 0 otherwise, i.e. a delta function?}
\subsection{Anisotropic XC energy}
\begin{equation}
    \E{AXC} = \sum_A (f^{\mu_A}_{XC}|\pmb{\mu}_A|^2 + f^{\Theta_A}_{XC}||\pmb{\Theta}_A||^2)
\end{equation}
Where $f^{\mu_A}_{XC}$ and $f^{\Theta_A}_{XC}$ are fitted values.

\noindent
\q{What norms are these, what are the formulas for calculating them?}
\newpage
\subsection{Dispersion Energy}
\begin{equation}
\begin{split}
  \E{disp}^{D4'} = &-\sum_{A>B} \sum_{n=6,8} s_n \frac{C_n^{AB} (q_A, CN^A_{cov}, q_B, CN^B_{cov})}{R^n_{AB}} f^{(n)}_{damp,BJ} (R_{AB}) \\
  &-s_9 \sum_{A>B>C} \frac{(3cos(\theta_{ABC})cos(\theta_{BCA})cos(\theta_{CAB})+1)C_9^{ABC}(CN_{cov}^A,CN_{cov}^B,CN_{cov}^C)}{(R_{AB} R_{AC} R_{BC})^3} \\
  &\times f^{(9)}_{damp,zero}(R_{AB},R_{AC},R_{BC}).
\end{split}
\end{equation}

\vspace{10pt}
\noindent
\(C_6^{AB}\) is the pairwise dipole-dipole dispersion coefficients calculated by numerical integration via the Casimir-Polder relation.
\begin{equation}
  C_6^{AB} = \frac{3}{\pi} \sum_{j} w_j \overline{\alpha}_A (i\omega_j, q_A, CN_{cov}^A)\overline{\alpha}_B (i\omega_j, q_B, CN_{cov}^B)
\end{equation}

\noindent
\(w_j\) are the integration weights, which are derived from a trapeziodal partitioning between the grid points \(j(j \in [1,23])\).

\vspace{10pt}
\noindent
\q{How are the integration weights calculated?}

\vspace{10pt}
\noindent
The isotropically averaged, dynamic dipole-dipole polarizabilites \(\overline{\alpha}\) at the \(j\)th imaginary frequency \(i\omega_j\) are obtained from the self-consistent D4 model; i.e., they are depending on the covalent coordination number and are also charge dependent.

\begin{equation}
  \overline{\alpha}_A(i\omega_j, q_A, CN_{cov}^A) = \sum_{r}^{N_{A,ref}} \xi_A^r (q_A, q_{A,r}) \overline{\alpha}_{A,r}(i\omega_j, q_{A,r}, CN_{cov}^{A,r}) W_A^r(CN_{cov}^A, CN_{cov}^{A,r})
\end{equation}

\vspace{10pt}
\noindent
\q{\(q_A\) is supposedly the atomic charge for atom A, but what is \(q_{A,r}\)?}

\vspace{10pt}
\noindent
The Gaussian weighting for each reference system is given by:
\begin{equation}
  W_A^r(CN_{cov}^A, CN_{cov}^{A,r}) = \sum_{j=1}^{N_{gauss}} \frac{1}{\mathcal{N}} \exp\left[-6j \cdot (CN_{cov}^A - CN_{cov}^{A,r})^2\right]
\end{equation}

with
\begin{equation}
  \sum_{r}^{N_{A,ref}} W_A^r(CN_{cov}^A, CN_{cov}^{A,r}) = 1
\end{equation}

\vspace{10pt}
\noindent
\(\mathcal{N}\) is a normalization constant.

\vspace{10pt}
\noindent
\q{Do you know what the constant \(\mathcal{N}\) is?}


\vspace{10pt}
\noindent
The number of Gaussian function per reference system \(N_{gauss}\) is mostly one, but equal to three for \(CN_{cov}^{A,r} = 0\) and reference systems with similar coordination number.

\noindent
The carge-dependency is included via the empirical scaling function \(\xi_A^r\).
\begin{equation}
  \xi_A^r(q_A, q_{A,r}) = \exp\left[3\left\{1-\exp\left[4\eta_A\left(1-\frac{Z_A^{eff} + q_{A,r}}{Z_A^{eff} + q_A}\right)\right]\right\}\right]
\end{equation}

\noindent
where \(\eta_A\) is the chemical hardness taken from ref 98.

\noindent
\(Z_A^{eff}\) is the effective nuclear charge of atom A, which has been determined by subtracting the number of core electrons represented by the def2-ECPs in the time-dependent DFT reference calculations.

\vspace{10pt}
\noindent
\(C_8^{AB}\) is calculated recursively from the lowest order \(C_6^{AB}\) coefficients.

\begin{equation}
  C_8^{AB} = 3C_6^{AB} \sqrt{\mathcal{Q}^A\mathcal{Q}^B}
\end{equation}

\begin{equation}
  \mathcal{Q}^A = s_{42} \sqrt{Z^A} \frac{\braket{r^4}^A}{\braket{r^2}^A}
\end{equation}


\vspace{10pt}
\noindent
\(\sqrt{Z^A}\) is the ad hoc nuclear charge dependent factor.

\vspace{10pt}
\noindent
\q{How do we calculate \(Z^A\)?}

\vspace{10pt}
\noindent
\(\braket{r4}\) and \(\braket{r2}\) are simple multipole-type expectation values derived from atomic densities which are averaged geometrically to get the pair coefficients.

\vspace{10pt}
\noindent
\q{How do we calculate \(r\)?}

\vspace{10pt}
\noindent
\q{Where do we find the value of \(s_{42}\)?}

\vspace{20pt}
\noindent
\(CN^A_{cov}\) is the covalent coordination number for atom A.

\vspace{10pt}
\noindent
\(q\) is the atomic charge, so \(q_A\) is the atomic charge for atom A.

\vspace{10pt}
\noindent
The damping and scaling parameters in the
dispersion model are:
\[
  a1 = 0.52 \quad|\quad a2 = 5.0 \quad|\quad s6 = 1.0 \quad|\quad s8 = 2.7 \quad|\quad s9 = 5.0
\]


\vspace{10pt}
\noindent
\(C_9^{ABC}\) is the triple-dipole constant\footnote{\label{dft-d}https://www.researchgate.net/publication/43347348\_A\_Consistent\_and\_Accurate\_Ab\_Initio\_Parametrization\_of\_Density\_Functional\_Dispersion\_Correction\_DFT-D\_for\_the\_94\_Elements\_H-Pu}:
\begin{equation}
  C_9^{ABC} = \frac{3}{\pi} \int_0^\infty \alpha^A(i\omega) \alpha^B(i\omega)\alpha^C(i\omega)d\omega
\end{equation}

\vspace{10pt}
\noindent
The three-body contribution is typically \(<5-10\%\) of \(E_{disp}\), so it is small enough that we can reasonably approximate the coefficients by a geometric mean as\footnoteref{dft-d}:

\begin{equation}
  C_9^{ABC} \approx -\sqrt{C_6^{AB} C_6^{AC} C_6^{BC}}
\end{equation}


\vspace{10pt}
\noindent
\q{Is it fine to just approximate \(C_9\)? Otherwise, what is \(i\omega\) and \(\alpha^A\)?}


\vspace{10pt}
\noindent
\(\theta_{ABC}\) is the angle between the two edges going from B to the other two atoms. \(\theta_{BCA}\) is the angle between the edges going from C to the other two and so on.


\vspace{10pt}
\noindent
BJ = Becke-Johnson

\begin{equation}
  f_n^{damp,BJ}(R_{AB}) = \frac{R_{AB}^n}{R_{AB}^n + (a_1 \cdot R_{AB}^{crit} + a_2)^6}
\end{equation}

\begin{equation}
  R_{AB}^{crit} = \sqrt{\frac{C_8^{AB}}{C_6^{AB}}}
\end{equation}


\begin{equation}
  f_9^{damp,zero}(R_{AB}, R_{AC}, R_{BC}) = \left(1 + 6 \left(\sqrt{\frac{R_{AB}^{crit} R_{BC}^{crit} R_{CA}^{crit}}{R_{AB} R_{BC} R_{CA}}}\right)^{16}\right)^{-1}
\end{equation}


\end{document}
