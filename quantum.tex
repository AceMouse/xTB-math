\documentclass{article}
\usepackage{amsmath}
\usepackage{braket}

\begin{document}
\section{GFN2-xTB $E^\Gamma$}
\begin{equation}
    E^\Gamma = \frac{1}{3}\sum_A\sum_{\mu\in A} (q_{A,\mu})^3\Gamma_{A,\mu}
\end{equation}
where $q_{A,\nu}=\sum_B\sum_{\nu\in B}P_{\mu\nu}S_{\mu\nu}$ is the partial charge of shell $\mu$ associated with atom $A$. $P, S$ are the density and overlap matrices. $\Gamma_{A,\mu} = \Gamma_A K_\mu$ is just the product of an element specific constant and a shell specific constant, for our purposes the element is always carbon and the shell is either the first or second in GFN2 thus we have 2 numbers $\Gamma_{\text{Carbon},1(2)}$ henceforth referred to as $\varGamma_{1(2)}$. 

\vspace{\baselineskip}
\noindent
Let us first rewrite the inner expression a bit given our new definition and knowledge of the atoms we are working with. 
\begin{equation}
    \sum_{\mu\in A} (q_{A,\mu})^3\Gamma_{A,\mu} = \sum_{\mu = 1}^2 (q_{A,\mu})^3\varGamma_{\mu}
\end{equation}

We now need a unitary which computes this function on a given state $\ket{q_{A,\mu}}\ket{\Gamma_\mu}\ket{acc} \to \ket{q_{A,\mu}}\ket{\Gamma_\mu}\ket{acc+(q_{A,\mu})^3\varGamma_{\mu}}$. 
Consider having access to the following addition (multiplication) unitary $\ket{A}\ket{B} \to \ket{A}\ket{A+(*)B}$ as well as the unitary taking $\ket{A} \to \ket{A^2}$. 
We can now apply the following to get our desired unitary.  
\begin{align}
    &\text{MULT}^\dagger_{x,y}\text{SQR}^\dagger_x\text{MULT}^\dagger_{x,y}\text{ADD}_{y,z}\text{MULT}_{x,y}\text{SQR}_x\text{MULT}_{x,y}\ket{q_{A,\mu}}_x\ket{\Gamma_\mu}_y\ket{acc}_z\\ 
    = &\text{MULT}^\dagger_{x,y}\text{SQR}^\dagger_x\text{MULT}^\dagger_{x,y}\text{ADD}_{y,z}\text{MULT}_{x,y}\text{SQR}_x\ket{q_{A,\mu}}_x\ket{\Gamma_\mu q_{A,\mu}}_y\ket{acc}_z\\ 
    = &\text{MULT}^\dagger_{x,y}\text{SQR}^\dagger_x\text{MULT}^\dagger_{x,y}\text{ADD}_{y,z}\text{MULT}_{x,y}\ket{(q_{A,\mu})^2}_x\ket{\Gamma_\mu q_{A,\mu}}_y\ket{acc}_z\\ 
    = &\text{MULT}^\dagger_{x,y}\text{SQR}^\dagger_x\text{MULT}^\dagger_{x,y}\text{ADD}_{y,z}\ket{(q_{A,\mu})^2}_x\ket{(q_{A,\mu})^3\Gamma_\mu}_y\ket{acc}_z\\ 
    = &\text{MULT}^\dagger_{x,y}\text{SQR}^\dagger_x\text{MULT}^\dagger_{x,y}\ket{(q_{A,\mu})^2}_x\ket{(q_{A,\mu})^3\Gamma_\mu }_y\ket{acc+(q_{A,\mu})^3\Gamma_\mu}_z\\ 
    = &\text{MULT}^\dagger_{x,y}\text{SQR}^\dagger_x\ket{(q_{A,\mu})^2}_x\ket{\Gamma_\mu q_{A,\mu}}_y\ket{acc+(q_{A,\mu})^3\Gamma_\mu}_z\\ 
    = &\text{MULT}^\dagger_{x,y}\ket{q_{A,\mu}}_x\ket{\Gamma_\mu q_{A,\mu}}_y\ket{acc+(q_{A,\mu})^3\Gamma_\mu}_z\\ 
    = &\ket{q_{A,\mu}}_x\ket{\Gamma_\mu}_y\ket{acc+(q_{A,\mu})^3\Gamma_\mu }_z\\ 
\end{align}
This circuit to compute the inner sum in $E^\Gamma$ could be called ${E_i^\Gamma}_{(x,y,z)}$. Let us say we are given the following state $$\bigotimes_A(\ket{q_{A,1}}_a\ket{q_{A,2}}_b)\otimes(\ket{\Gamma_1}_c\ket{\Gamma_2}_d\ket{0}_e)$$ 
we can now apply $\prod_{a} {E_i^\Gamma}_{(a,c,e)}\prod_{b} {E_i^\Gamma}_{(b,d,e)}$ to get 
$$\bigotimes_A(\ket{q_{A,1}}_a\ket{q_{A,2}}_b)\otimes(\ket{\Gamma_1}_c\ket{\Gamma_2}_d\ket{0}_e) \to \bigotimes_A(\ket{q_{A,1}}_a\ket{q_{A,2}}_b)\otimes(\ket{\Gamma_1}_c\ket{\Gamma_2}_d\ket{E^\Gamma}_e)$$




\end{document}
