\documentclass{article}
\usepackage{amsmath}
\usepackage{braket}
\usepackage{graphicx}
\begin{document}
\section{GFN2-xTB $E^\Gamma$}
\subsection{Quantum Digital Arithmetic}
\begin{equation}
    E^\Gamma = \frac{1}{3}\sum_A\sum_{\mu\in A} (q_{A,\mu})^3\Gamma_{A,\mu}
\end{equation}
where $q_{A,\nu}=\sum_B\sum_{\nu\in B}P_{\mu\nu}S_{\mu\nu}$ is the partial charge of shell $\mu$ associated with atom $A$. $P, S$ are the density and overlap matrices. $\Gamma_{A,\mu} = \Gamma_A K_\mu$ is just the product of an element specific constant and a shell specific constant, for our purposes the element is always carbon and the shell is either the first or second in GFN2 thus we have 2 numbers $\Gamma_{\text{Carbon},1(2)}$ henceforth referred to as $\varGamma_{1(2)}$. 

\vspace{\baselineskip}
\noindent
Let us first rewrite the inner expression a bit given our new definition and knowledge of the atoms we are working with. 
\begin{equation}
    \sum_{\mu\in A} (q_{A,\mu})^3\Gamma_{A,\mu} = \sum_{\mu = 1}^2 (q_{A,\mu})^3\varGamma_{\mu}
\end{equation}

We now need a unitary which computes this function on a given state $\ket{q_{A,\mu}}\ket{\varGamma_\mu}\ket{acc} \to \ket{q_{A,\mu}}\ket{\varGamma_\mu}\ket{acc+(q_{A,\mu})^3\varGamma_{\mu}}$. 
Consider having access to the following addition (multiplication) unitary $\ket{A}\ket{B} \to \ket{A}\ket{A+(*)B}$ as well as the unitary taking $\ket{A} \to \ket{A^2}$. 
We can now apply the following to get our desired unitary.  
\begin{align}
    &\text{MULT}^\dagger_{x,y}\text{SQR}^\dagger_x\text{MULT}^\dagger_{x,y}\text{ADD}_{y,z}\text{MULT}_{x,y}\text{SQR}_x\text{MULT}_{x,y}\ket{q_{A,\mu}}_x\ket{\varGamma_\mu}_y\ket{acc}_z\\ 
    = &\text{MULT}^\dagger_{x,y}\text{SQR}^\dagger_x\text{MULT}^\dagger_{x,y}\text{ADD}_{y,z}\text{MULT}_{x,y}\text{SQR}_x\ket{q_{A,\mu}}_x\ket{\varGamma_\mu q_{A,\mu}}_y\ket{acc}_z\\ 
    = &\text{MULT}^\dagger_{x,y}\text{SQR}^\dagger_x\text{MULT}^\dagger_{x,y}\text{ADD}_{y,z}\text{MULT}_{x,y}\ket{(q_{A,\mu})^2}_x\ket{\varGamma_\mu q_{A,\mu}}_y\ket{acc}_z\\ 
    = &\text{MULT}^\dagger_{x,y}\text{SQR}^\dagger_x\text{MULT}^\dagger_{x,y}\text{ADD}_{y,z}\ket{(q_{A,\mu})^2}_x\ket{(q_{A,\mu})^3\varGamma_\mu}_y\ket{acc}_z\\ 
    = &\text{MULT}^\dagger_{x,y}\text{SQR}^\dagger_x\text{MULT}^\dagger_{x,y}\ket{(q_{A,\mu})^2}_x\ket{(q_{A,\mu})^3\varGamma_\mu }_y\ket{acc+(q_{A,\mu})^3\varGamma_\mu}_z\\ 
    = &\text{MULT}^\dagger_{x,y}\text{SQR}^\dagger_x\ket{(q_{A,\mu})^2}_x\ket{\varGamma_\mu q_{A,\mu}}_y\ket{acc+(q_{A,\mu})^3\varGamma_\mu}_z\\ 
    = &\text{MULT}^\dagger_{x,y}\ket{q_{A,\mu}}_x\ket{\varGamma_\mu q_{A,\mu}}_y\ket{acc+(q_{A,\mu})^3\varGamma_\mu}_z\\ 
    = &\ket{q_{A,\mu}}_x\ket{\varGamma_\mu}_y\ket{acc+(q_{A,\mu})^3\varGamma_\mu }_z\\ 
\end{align}
This circuit to compute the inner sum in $E^\Gamma$ could be called ${E_i^\Gamma}_{(x,y,z)}$. Let us say we are given the following state 
\begin{equation}
\ket{0}_e\bigotimes_{A_n}^{\#A}\left(\ket{q_{A_n,1}}_{a_n}\ket{q_{A_n,2}}_{b_n}\right)\ket{\varGamma_1}_c\ket{\varGamma_2}_d 
\end{equation}
We can now apply $\prod_{a_n}^{\#A} {E_i^\Gamma}_{(a_n,c,e)}\prod_{b_n}^{\#A} {E_i^\Gamma}_{(b_n,d,e)}$ to get 
\begin{equation}
\resizebox{.9\hsize}{!}{$\ket{0}_e\bigotimes_{A_n}^{\#A}\left(\ket{q_{A_n,1}}_{a_n}\ket{q_{A_n,2}}_{b_n}\right)\ket{\varGamma_1}_c\ket{\varGamma_2}_d \to \ket{E^\Gamma}_e\bigotimes_{A_n}^{\#A}\left(\ket{q_{A_n,1}}_{a_n}\ket{q_{A_n,2}}_{b_n}\right)\ket{\varGamma_1}_c\ket{\varGamma_2}_d$}
\end{equation}

\subsection{Sampling using Quantum Amplitude Arithmetic}
Assume that we are given a superposition of all the fullerenes in an isomer-space with different charges which we then apply the algorithm on. Each isomer is also given a canonical ID as part of the state. We now have computed the $E^\Gamma$ energies for every isomer. However we can only sample once! Let us say that we are interested in the isomers with the lowest energies. We then would like the probability of sampling an isomer to be proportional to $E^\Gamma$. We can achieve this using Quantum Amplitude Arithmetic, not to be confused with Quantum Amplitude Amplification, both shortened as QAA but in this writing as QA-Arithmetic and QA-Amplification. 


\vspace{\baselineskip}
The paper on QA-Arithmetic uses the introduced addition and multiplication primitives to construct a circuit which transforms the state $\ket{x}_D\ket{0}_C\ket{0}_W \to \frac{1}{2}\frac{x}{2^n}\ket{x}_D\ket{0}_C\ket{1}_W+\alpha\ket{\omega}_{D \otimes C\otimes W}$ where $\alpha$ is some normalization factor, and $\ket{\omega}$ is some state with no overlap with the state containing all 0's in the C register and 1 in the W register.


\vspace{\baselineskip}
When using this circuit we can add to our state the C and W registers and treat the $e$ register containing our resulting $E^\Gamma$ term as the D register. Let us take a look at that.
We
\begin{equation}
   \resizebox{.9\hsize}{!}{$\begin{split}
        \sum_{ID\in \text{isomers}}&\left [ \ket{ID}\ket{E^\Gamma_{ID}}_e\ket{0}_C\ket{0}_W\bigotimes_{A_n}^{\#A}\left ( \ket{q_{A_n,1}}_{a_n}\ket{q_{A_n,2}}_{b_n}\right ) \right ] \ket{\varGamma_1}_c\ket{\varGamma_2}_d \to\\ 
        \sum_{ID\in \text{isomers}}&\left [ \ket{ID}\left(\frac{1}{2}\frac{E^\Gamma_{ID}}{2^n}\ket{E^\Gamma_{ID}}_e\ket{0}_C\ket{1}_W+\alpha_{ID}\ket{\omega_{ID}}\right)\bigotimes_{A_n}^{\#A}\left ( \ket{q_{A_n,1}}_{a_n}\ket{q_{A_n,2}}_{b_n}\right ) \right ] \ket{\varGamma_1}_c\ket{\varGamma_2}_d\\ 
   \end{split}$}
\end{equation}
As we don't care about the case where we get $\omega$ we can collect the terms a bit:
\begin{equation}
    \resizebox{.9\hsize}{!}{$\alpha_{tot}\ket{\omega_{tot}} + (1-\alpha_{tot})\sum_{ID\in \text{isomers}}\left [ \frac{1}{2}\frac{E^\Gamma_{ID}}{2^n}\ket{ID}\ket{E^\Gamma_{ID}}_e\ket{0}_C\ket{1}_W\bigotimes_{A_n}^{\#A}\left ( \ket{q_{A_n,1}}_{a_n}\ket{q_{A_n,2}}_{b_n}\right ) \right ] \ket{\varGamma_1}_c\ket{\varGamma_2}_d$}\\ 
\end{equation}
If we now sample from this superposition and postselect for C = 0 and W = 1 we are more likely to sample a low energy fullerene. 

\subsection{An alternative circuit using exclusively Quantum Amplitude Arithmetic.}
An alternative strategy would be to go all in on QA-Arithmetic and do all the arithmetic in the amplitudes. 
Here we would encode a molecule as follows
\begin{equation}
    \ket{ID}\sum_{\mu\in\{0,1\}}\ket{\varGamma_\mu}_\Gamma\sum_{A_n}^{\#A}\ket{q_{A_n,\mu}}_Q\ket{0}_{C}\ket{0}_{W}
\end{equation}
Where $C = C_1\otimes C_2\otimes C_3\otimes C_4$ and $W = W_1\otimes W_2\otimes W_3\otimes W_4$
We can now apply the QA-Arithmetic papers transformation 4 times. This would be on the $(\Gamma,C_1,W_1), (Q,C_2,W_2), (Q,C_3,W_3), (Q,C_4,W_4)$ registers.
This would, if we collect all the $\omega$ terms, yield
\begin{equation}
    \alpha_{ID}\ket{\omega_{ID}}+\frac{1}{2^4}\ket{ID}\sum_{\mu\in\{0,1\}}\frac{\varGamma_\mu}{2^{n}}\ket{\varGamma_\mu}_\Gamma\sum_{A_n}^{\#A}\frac{q_{A_n,\mu}^3}{2^{n^3}}\ket{q_{A_n,\mu}}_Q\ket{0}_{C}\ket{1}_{W}
\end{equation}

If we apply this to a superposition of such molecule encodings and postselect on C = 0 and W = 1 we would sample a molecule $m_{ID}$ with probability $\propto
    (1-\alpha_{tot})\sum_{\mu\in\{0,1\}}\sum_{A_n}^{\#A}\frac{\varGamma_\mu}{2^{n}}\frac{q_{A_n,\mu}^3}{2^{n^3}} =  (1-\alpha_{tot})\frac{E^\Gamma_{ID}}{2^{n^4}}\propto E^\Gamma_{ID}
$
We can then compute the energy classically. 
\subsection{Discussion}
QA-Amplification might be possible since we have a very clear "good" state in both algorithms. This would reduce the need for postselection.




\end{document}
