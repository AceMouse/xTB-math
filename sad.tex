\subsection{SAD - Superposition of Atomic Densities}

The superposition of atomic densities(SAD) is an approach to obtain a good approximation of a collection of atoms, to be used as an initial guess for solving the self-consistent field(SCF) equation.

As originally implemented in DISCO, the molecular electron density can be obtained by adding the densities of all the constituting atoms.


%https://pyscf.org/user/scf.html#initial-guess
%https://sci-hub.box/10.1002/jcc.20393

This is how we get the density matrix for an isolated atom? equation 15 from: (https://sci-hub.box/10.1002/jcc.540030314)
\begin{equation}
  D_{ij} = \sum_a^{occ} c_{ia} c_{ja}
\end{equation}

To get the coefficients we need to solve SCF for each atom? this is supposedly cheap, but idk how to do it. (https://sci-hub.box/10.1002/jcc.20393)
Though the math for Direct SCF Approach is given in this paper at equation 10: (https://sci-hub.box/10.1002/jcc.540030314). This is probably how.

The SAD method is then the sum of all of these?

Equation 3 in the GFN2 paper talks about "superposition of (neutral) atomic reference densities". Is this relevant?

Direct SCF Approach
\begin{equation}
\begin{split}
  \Delta F_{ab} = &(c_{ia}c_{jb} + c_{ja}c_{ib})\\
  &\Delta F_{ij} + (c_{ia}c_{kb} + c_{ka}c_{ib})\\
  &\Delta F_{ik} + (c_{ia}c_{lb} + c_{la}c_{ib})\\
  &\Delta F_{il} + (c_{ja}c_{kb} + c_{ka}c_{jb})\\
  &\Delta F_{jk} + (c_{ja}c_{lb} + c_{la}c_{jb})\\
  &\Delta F_{jl} + (c_{ka}c_{lb} + c_{la}c_{kb}) \Delta F_{kl}\\
  &= l_{ijkl}(4E_{ij}^{ab}D_{kl} + 4D_{ij}E_{kl}^{ab} - E_{ik}^{ab}D_{jl} - D_{ik}E_{jl}^{ab} - E_{il}^{ab}D_{jk} - D_{il}E_{jk}^{ab})
\end{split}
\end{equation}
where
\begin{equation}
  E_{ij}^{ab} = c_{ia}c_{jb} + c_{ja}c_{ib}
\end{equation}

