\documentclass{article}
\usepackage{amsmath}
\usepackage{braket}

\makeatletter
\newcommand{\q}[1]{\textbf{QUESTION: #1}}
\newcommand\E{\@ifnextchar[{\@with}{\@without}}
\def\@with[#1]#2{E_{#2}^{(#1)}}
\def\@without#1{E_{#1}}
\makeatother

\makeatletter
\newcommand\footnoteref[1]{\protected@xdef\@thefnmark{\ref{#1}}\@footnotemark}
\makeatother

\begin{document}

\section{Gaussian functions per reference system}
\noindent
The Gaussian weighting for each reference system is given by:
\begin{equation}
  W_A^r(CN_{cov}^A, CN_{cov}^{A,r}) = \sum_{j=1}^{N_{gauss}} \frac{1}{\mathcal{N}} \exp\left[-6j \cdot (CN_{cov}^A - CN_{cov}^{A,r})^2\right]
\end{equation}

\noindent
with
\begin{equation}
  \sum_{r}^{N_{A,ref}} W_A^r(CN_{cov}^A, CN_{cov}^{A,r}) = 1
\end{equation}

\vspace{10pt}
\noindent
\(\mathcal{N}\) is a normalization constant.

\noindent
The number of Gaussian functions per reference system \(N_{gauss}\) is mostly one, but equal to three for \(CN_{cov}^{A,r} = 0\) and reference systems with similar coordination number.

\q{How do we obtain the number of gaussian functions per reference system (\(N_{gauss})\)?}

\q{How or where do we get the normalization constant?}



\section{SAD - Superposition of Atomic Densities}

The superposition of atomic densities(SAD) is an approach to obtain a good approximation of a collection of atoms, to be used as an initial guess for solving the self-consistent field(SCF) equation.

As originally implemented in DISCO, the molecular electron density can be obtained by adding the densities of all the constituting atoms.


%https://pyscf.org/user/scf.html#initial-guess
%https://sci-hub.box/10.1002/jcc.20393

This is how we get the density matrix for an isolated atom?

\begin{equation}
  \rho_0 = \sum_A \rho_0^A
\end{equation}
\begin{equation}
  D_{ij} = \sum_a^{occ} c_{ia} c_{ja}
\end{equation}

To get the coefficients we need to solve SCF for each atom.
The SAD method is then the sum of all of these.

\vspace{10pt}
\noindent
\q{How can we get the initial guess for the Fock matrix and Coefficient matrices needed to run SCF on the individual atoms?}

Direct SCF Approach
\begin{equation}
\begin{split}
  \Delta F_{ab} = &(c_{ia}c_{jb} + c_{ja}c_{ib})\\
  &\Delta F_{ij} + (c_{ia}c_{kb} + c_{ka}c_{ib})\\
  &\Delta F_{ik} + (c_{ia}c_{lb} + c_{la}c_{ib})\\
  &\Delta F_{il} + (c_{ja}c_{kb} + c_{ka}c_{jb})\\
  &\Delta F_{jk} + (c_{ja}c_{lb} + c_{la}c_{jb})\\
  &\Delta F_{jl} + (c_{ka}c_{lb} + c_{la}c_{kb}) \Delta F_{kl}\\
  &= l_{ijkl}(4E_{ij}^{ab}D_{kl} + 4D_{ij}E_{kl}^{ab} - E_{ik}^{ab}D_{jl} - D_{ik}E_{jl}^{ab} - E_{il}^{ab}D_{jk} - D_{il}E_{jk}^{ab})
\end{split}
\end{equation}
where
\begin{equation}
  E_{ij}^{ab} = c_{ia}c_{jb} + c_{ja}c_{ib}
\end{equation}



\section{Questions for Albert}

\q{Does the xTB program compute the self-consistent field(SCF) or is that part of an external project?}

\vspace{10pt}
\noindent
\q{How do you get your initial guess for the Fock matrix and Coefficient matrices needed to run SCF on the individual atoms?}

\vspace{10pt}
\noindent
\q{Can you shed some light on where and how the xTB program computes the initial guess?}

\vspace{10pt}
\noindent
\q{How and where is the overlap matrix computed? Which terms are needed?}


\end{document}
