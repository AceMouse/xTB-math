\section{Fock Matrix for GFN2-xTB}
\begin{equation}
\begin{split}
    \mn{F^{GFN2-xTB}} =  \mn{H^{EHT}} + \mn{F^{IES+IXC}} + &\mn{F^{AES}}+\mn{F^{AXC}}+\mn{F^{D4}}, \\&\forall \mu \in A, \nu \in B
\end{split}
\end{equation}
\subsection{Isotropic Electrostatic and Exchange-correlation contribution}
\begin{equation}
\begin{split}
    \mn{F^{IES+IXC}} = &- \frac{1}{2}\mn{S}\sum_C\sum_{l''}(\gamma_{AC,ll''}+\gamma_{BC,l'l''})q_{C,l''}\\
    &-\frac{1}{2}\mn{S}(q_{A,l}^2\Gamma_{A,l}+q_{B,l'}^2\Gamma_{B,l'})
\end{split}
\end{equation}
$l,l',l''$ being the angular momenta of the orbitals $\mu, \nu$ and each of C's orbitals. 
\begin{equation}
    \Gamma_{A,l} = K^\Gamma_l\Gamma_A
\end{equation}
$K^\Gamma_l$ is a shell specific constant common for all elements and $\Gamma_A$ is an element specific constant. 
\begin{align}
    \gamma_{AB,ll'} &= \frac{1}{\sqrt{R_{AB}^2+\eta^{-2}_{AB,ll'}}}\\
    \eta_{AB,ll'}&=\frac{1}{2}\left[\eta_A(1+k_A^l)+\eta_B(1+k_B^{l'})\right]
\end{align}
$q_l$ is a partial Mulliken charge. $\eta_A$ and $\eta_B$ are element-specific fit parameters, while $k_A^l$ and $k_B^{l'}$ are element-specific scaling factors for the individual shells ($k_A^l=0$ when $l=0$).
\begin{align}
    GAP_A &= \sum_{l \in A} q_{A,l}\\
    q_{A,l} &= \sum_{l'\in B}P_{ll'}S_{ll'} = GOP_l
\end{align}


\subsection{Dispersion contribution}
\begin{align}
    \mn{F^{D4}} &= -\frac{1}{2}\mn{S}(d_A+d_B), \forall \mu \in A, \nu \in B\\
\begin{split}
    d_A &= \sum_r^{N_{A,ref}} \frac{\partial \xi^r_A(q_A,q_{A,r})}{\partial q_A}\sum_B\sum_s^{N_{B,ref}}\sum_{n=6,8}\\
    &\quad\quad W_A^r(CN^A_{cov},CN^{A,r}_{cov})W_B^s(CN^B_{cov},CN^{B,s}_{cov})\xi^s_B(q_B,q_{B,s})\times\\
    &\quad\quad s_n\frac{C^{AB,ref}_n}{R_{AB}^n}f_n^{damp,BJ}(R_{AB}) 
\end{split}
\end{align}

The dispersion coefficient for two reference atoms \(C_n^{AB,\text{ref}}\) is evaluated at the reference points, i.e., for \(q_A = q_r\), \(q_B = q_s\), \(CN_{\text{cov}}^A = CN_{\text{cov}}^r\), and \(CN_{\text{cov}}^B = CN_{\text{cov}}^s\).

\vspace{10pt}
\noindent
The Gaussian weighting for each reference system is given by:
\begin{equation}
  W_A^r(CN_{cov}^A, CN_{cov}^{A,r}) = \sum_{j=1}^{N_{gauss}} \frac{1}{\mathcal{N}} \exp\left[-6j \cdot (CN_{cov}^A - CN_{cov}^{A,r})^2\right]
\end{equation}

with
\begin{equation}
  \sum_{r}^{N_{A,ref}} W_A^r(CN_{cov}^A, CN_{cov}^{A,r}) = 1
\end{equation}

\vspace{10pt}
\noindent
\(\mathcal{N}\) is a normalization constant.

\noindent
The number of Gaussian function per reference system \(N_{gauss}\) is mostly one, but equal to three for \(CN_{cov}^{A,r} = 0\) and reference systems with similar coordination number.


\vspace{10pt}
\noindent
\(C_6^{AB}\) is the pairwise dipole-dipole dispersion coefficients calculated by numerical integration via the Casimir-Polder relation.
\begin{equation}
  C_6^{AB} = \frac{3}{\pi} \sum_{j} w_j \overline{\alpha}_A (i\omega_j, q_A, CN_{cov}^A)\overline{\alpha}_B (i\omega_j, q_B, CN_{cov}^B)
\end{equation}

\noindent
\(w_j\) are the integration weights, which are derived from a trapeziodal partitioning between the grid points \(j(j \in [1,23])\).

\noindent
The isotropically averaged, dynamic dipole-dipole polarizabilites \(\overline{\alpha}\) at the \(j\)th imaginary frequency \(i\omega_j\) are obtained from the self-consistent D4 model; i.e., they are depending on the covalent coordination number and are also charge dependent.

\begin{equation}
  \overline{\alpha}_A(i\omega_j, q_A, CN_{cov}^A) = \sum_{r}^{N_{A,ref}} \xi_A^r (q_A, q_{A,r}) \overline{\alpha}_{A,r}(i\omega_j, q_{A,r}, CN_{cov}^{A,r}) W_A^r(CN_{cov}^A, CN_{cov}^{A,r})
\end{equation}

\noindent
The charge-dependency is included via the empirical scaling function \(\xi_A^r\).
\begin{equation}
  \xi_A^r(q_A, q_{A,r}) = \exp\left[3\left\{1-\exp\left[4\eta_A\left(1-\frac{Z_A^{eff} + q_{A,r}}{Z_A^{eff} + q_A}\right)\right]\right\}\right]
\end{equation}

\noindent
where \(\eta_A\) is the chemical hardness taken from ref 98.

\noindent
\(Z_A^{eff}\) is the effective nuclear charge of atom A, which has been determined by subtracting the number of core electrons represented by the def2-ECPs in the time-dependent DFT reference calculations.



\vspace{10pt}
\noindent
\(C_8^{AB}\) is calculated recursively from the lowest order \(C_6^{AB}\) coefficients.

\begin{equation}
  C_8^{AB} = 3C_6^{AB} \sqrt{\mathcal{Q}^A\mathcal{Q}^B}
\end{equation}

\begin{equation}
  \mathcal{Q}^A = s_{42} \sqrt{Z^A} \frac{\braket{r^4}^A}{\braket{r^2}^A}
\end{equation}


\vspace{10pt}
\noindent
\(\sqrt{Z^A}\) is the ad hoc nuclear charge dependent factor.

From the original xTB program we can see that \(s_{42}\) is \(0.5\), and \(Z^A\) is the atomic number of A.

\begin{equation}
  \sqrt{0.5 * (\frac{r^4}{r^2} * \sqrt{Z^A})}
\end{equation}

\(\braket{r4}\) and \(\braket{r2}\) are simple multipole-type expectation values derived from atomic densities which are averaged geometrically to get the pair coefficients.

\vspace{20pt}
\noindent
\(CN^A_{cov}\) is the covalent coordination number for atom A.

\vspace{10pt}
\noindent
\(q\) is the atomic charge, so \(q_A\) is the atomic charge for atom A.

\vspace{10pt}
\noindent
The scaling parameters in the dispersion model are:
\[
  s6 = 1.0 \quad|\quad s8 = 2.7
\]

\vspace{10pt}
\noindent
BJ = Becke-Johnson

\begin{equation}
  f_n^{damp,BJ}(R_{AB}) = \frac{R_{AB}^n}{R_{AB}^n + (a_1 \cdot R_{AB}^{crit} + a_2)^6}
\end{equation}

\begin{equation}
  R_{AB}^{crit} = \sqrt{\frac{C_8^{AB}}{C_6^{AB}}}
\end{equation}


\begin{equation}
  f_9^{damp,zero}(R_{AB}, R_{AC}, R_{BC}) = \left(1 + 6 \left(\sqrt{\frac{R_{AB}^{crit} R_{BC}^{crit} R_{CA}^{crit}}{R_{AB} R_{BC} R_{CA}}}\right)^{16}\right)^{-1}
\end{equation}
