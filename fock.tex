\section{Fock Matrix for GFN2-xTB}
\begin{equation}
\begin{split}
    \mn{F^{GFN2-xTB}} =  \mn{H^{EHT}} + \mn{F^{IES+IXC}} + &\mn{F^{AES}}+\mn{F^{AXC}}+\mn{F^{D4}}, \\&\forall \mu \in A, \nu \in B
\end{split}
\end{equation}
\subsection{Isotropic Electrostatic and Exchange-correlation contribution}
\begin{equation}
\begin{split}
    \mn{F^{IES+IXC}} = &- \frac{1}{2}\mn{S}\sum_C\sum_{l''}(\gamma_{AC,ll''}+\gamma_{BC,l'l''})q_{C,l''}\\
    &-\frac{1}{2}\mn{S}(q_{A,l}^2\Gamma_{A,l}+q_{B,l'}^2\Gamma_{B,l'})
\end{split}
\end{equation}
$l,l',l''$ being the angular momenta of the orbitals $\mu, \nu$ and each of C's orbitals. 
\begin{equation}
    \Gamma_{A,l} = K^\Gamma_l\Gamma_A
\end{equation}
$K^\Gamma_l$ is a shell specific constant common for all elements and $\Gamma_A$ is an element specific constant. 
\begin{align}
    \gamma_{AB,ll'} &= \frac{1}{\sqrt{R_{AB}^2+\eta^{-2}_{AB,ll'}}}\\
    \eta_{AB,ll'}&=\frac{1}{2}\left[\eta_A(1+k_A^l)+\eta_B(1+k_B^{l'})\right]
\end{align}
$q_l$ is a partial Mulliken charge. $\eta_A$ and $\eta_B$ are element-specific fit parameters, while $k_A^l$ and $k_B^{l'}$ are element-specific scaling factors for the individual shells ($k_A^l=0$ when $l=0$).
\begin{align}
    GAP_A &= \sum_{l \in A} q_{A,l}\\
    q_{A,l} &= \sum_{l'\in B}P_{ll'}S_{ll'} = GOP_l
\end{align}

\subsection{Anisotropic Electrostatic and Exchange-correlation contribution}
\begin{align}
    \begin{split}
    \mn{F^{AES}}+\mn{F^{AXC}} &= \frac{1}{2}\mn{S}\left[V_S(\pmb{R}_B)+ V_S(\pmb{R}_C)\right]\\
    &+ \frac{1}{2}\mn{\pmb{D}^T}\left[\pmb{V}_D(\pmb{R}_B)+\pmb{V}_D(\pmb{R}_C)\right]\\
    &+ \frac{1}{2} \sum_{\alpha,\beta\in\{x,y,z\}} \mn{Q^{\alpha\beta}}\left[V_Q^{\alpha\beta}(\pmb{R}_B)+ V_Q^{\alpha\beta}(\pmb{R}_C)\right]
    \end{split}\\
    \mn{\pmb{D}^T} &=
    \begin{pmatrix}
        \mn{D^x} &
        \mn{D^y} &
        \mn{D^z}\\
    \end{pmatrix}\\
\end{align}
\begin{align}
    \begin{split}
        V_S(\pmb{R}_C)& = \sum_A \Biggl\{
        \pmb{R}_C^T\biggl[
            f_5(R_{AC})\pmb{\mu}_AR^2_{AC} - 
            \pmb{R}_{AC}3f_5(R_{AC})(\pmb{\mu}_A^T\pmb{R}^2_{AC}) \\&- 
            f_3(R_{AC})q_A\pmb{R}_{AC}
        \biggr] 
    - f_5(R_{AC})\pmb{R}^T_{AC}\pmb{\Theta}_A\pmb{R}_{AC}
          - f_3(R_{AC})\pmb{\mu}^T_{A}\pmb{R}_{AC}\\
          &+q_Af_5(R_{AC})\frac{1}{2}\pmb{R}_C^2\pmb{R}_{AC}^2 
          - \frac{3}{2}q_Af_5(R_{AC})\sum_{\alpha\beta}\alpha_{AB}\beta_{AB}\alpha_{C}\beta_{C}\Biggr\}\\
      &+2f_{XC}^{\mu_C}\pmb R^T_C\pmb\mu_C - f_{XC}^{\Theta_C}\pmb R^T_C\biggl[3\pmb\Theta_C-\text{Tr}(\pmb\Theta_C)\pmb I\biggr]\pmb R_C
    \end{split}
\end{align}
\q{Should this not be $\pmb{R}_C^{2,T}$, in line 3, term 1?}

\begin{align}
    \begin{split}
        V_D(\pmb{R}_C)
            &= \sum_A \Biggl[
                \pmb{R}_{AC}3f_5(R_{AC})(\pmb{\mu}_A^T\pmb{R}_{AC}) -
                f_5(R_{AC})\pmb{\mu}_AR^2_{AC} + 
                f_3(R_{AC})q_A\pmb{R}_{AC}\\
            &-q_Af_5(R_{AC})\pmb{R}_CR_{AC}^2 + 
                3q_Af_5(R_{AC})\pmb{R}_{AC}\sum_\alpha\alpha_C\alpha_{AC}\Biggr]\\
      &-2f_{XC}^{\mu_C}\pmb\mu_C - 2f_{XC}^{\Theta_C}\biggl[3\pmb\Theta_C-\text{Tr}(\pmb\Theta_C)\pmb I\biggr]\pmb R_C
    \end{split}
\end{align}
\begin{align}
    \begin{split}
        V_Q^{\alpha\beta}(\pmb{R}_C)
            &= -\sum_A q_Af_5(R_{AC})\left[          
                \frac{3}{2}\alpha_{AC}\beta_{AC}- \frac{1}{2}R_{AB}^2 % TODO: is AB correct here!?
                \right]\\
        &-f_{XC}^{\Theta_C}\left[3\pmb\Theta_C^{\alpha\beta}-\delta_{\alpha\beta}\sum_\alpha\pmb\Theta_C^{\alpha\alpha}\right]
    \end{split}
\end{align}

$\pmb{\mu}_A$ is the cumulative atomic dipole moment of atom A and $\pmb{\Theta}_A$ is the corresponding traceless quadrupole moment. Traceless simply means that the sum of the diagonal elements is 0. The curly braces and brackets are used in the same way as normal parenthesis for showing order of operations. $q_A$ is the atomic charge of atom A. 
\begin{align}
    \Theta_A^{\alpha\beta} &= \frac{3}{2} \theta_A^{\alpha\beta} - \frac{\delta_{\alpha\beta}}{2} \left( \theta_A^{xx} + \theta_A^{yy} + \theta_A^{zz} \right)\\
    \theta_A^{\alpha\beta} &= \sum_{l' \in A} \sum_{l} P_{l} \left( \alpha_A D_{ll'}^{\beta} + \beta_A D_{ll'}^{\alpha} - \alpha_A \beta_A S_{ll'} - Q_{ll'}^{\alpha\beta} \right)\\
    q_A &= Z_A - GAP_A\\
    \mu_A^{\alpha} &= \sum_{l' \in A} \sum_{l} P_{l'l} \left( \alpha_A S_{l'l} - D_{l'l}^{\alpha} \right)\\
    D_{ll'}^{\alpha} &= \braket{ \phi_{l} | \alpha_i | \phi_{l'}} = \alpha_i S_{ll'}\\
    Q_{ll'}^{\alpha\beta} &= \braket{ \phi_{l} | \alpha_i\beta_i | \phi_{l'}} = \beta_i D_{ll'}^\alpha
\end{align}
$\alpha$ and $\beta$ are Cartesian components labled $(x,y,z)^T$ with atom A being centered in $\pmb{R}_A = (x_i,y_i,z_i)^T$ where i is a form of pointer/label dereferencing. $\delta_{\alpha\beta}$ is just the delta function, i.e is is 1 if $\alpha$ and $\beta$ are the same label and 0 otherwise, this serves to include the term only for the diagonal. 

\begin{equation}
    \pmb{\Theta}_A = 
    \begin{pmatrix}
        \Theta_A^{xx} & \Theta_A^{xy} & \Theta_A^{xz}\\
        \Theta_A^{yx} & \Theta_A^{yy} & \Theta_A^{yz}\\
        \Theta_A^{zx} & \Theta_A^{zy} & \Theta_A^{zz}\\
    \end{pmatrix}
\end{equation}
\begin{equation}
    \pmb{\mu}_A = 
    \begin{pmatrix}
        \mu_A^{x}&
        \mu_A^{y}&
        \mu_A^{z}\\
    \end{pmatrix}^T
\end{equation}
\begin{align}
    \pmb{R}_{AB} &= \pmb{R}_A-\pmb{R}_B\\
    R_{AB} &= \sqrt{(\pmb{R}^x_{AB})^2+(\pmb{R}^y_{AB})^2+(\pmb{R}^z_{AB})^2}
\end{align}


\begin{align}
    f_n(R_{AB}) &= \frac{f_{damp}(a_n,R_{AB})}{R_{AB}^n}=\frac{1}{R_{AB}^n}\frac{1}{1+6\left(\frac{R_0^{AB}}{R_{AB}}\right)^{a_n}}\\
    R_0^{AB} &= 0.5 ({R^A_0}'+ {R^B_0}')\\
    {R^A_0}' &= \begin{cases}R^A_0 + \frac{R_{max}-R^A_0}{1+exp[-4(CN_A'-N_{val}-\Delta_{val})]} & \text{if }N_{val}\text{ is given}\\5.0 \text{ bohrs} & \text{otherwise}\end{cases}\\
        R_{max} &= 5.0 \text{ bohrs}\\
    \Delta_{val} &= 1.2
\end{align}
$R_0^A$ is a fitted value for 12 elements and 5.0 for the rest. $a_n$ are adjusted global parameters. 
Where $f^{\mu_A}_{XC}$ and $f^{\Theta_A}_{XC}$ are fitted values. 
\subsection{Dispersion contribution}
\begin{align}
    \mn{F^{D4}} &= -\frac{1}{2}\mn{S}(d_A+d_B), \forall \mu \in A, \nu \in B\\
\begin{split}
    d_A &= \sum_r^{N_{A,ref}} \frac{\partial \xi^r_A(q_A,q_{A,r})}{\partial q_A}\sum_B\sum_s^{N_{B,ref}}\sum_{n=6,8}\\
    &\quad\quad W_A^r(CN^A_{cov},CN^{A,r}_{cov})W_B^s(CN^B_{cov},CN^{B,s}_{cov})\xi^s_B(q_B,q_{B,s})\times\\
    &\quad\quad s_n\frac{C^{AB,ref}_n}{R_{AB}^n}f_n^{damp,BJ}(R_{AB}) 
\end{split}
\end{align}

The dispersion coefficient for two reference atoms \(C_n^{AB,\text{ref}}\) is evaluated at the reference points, i.e., for \(q_A = q_r\), \(q_B = q_s\), \(CN_{\text{cov}}^A = CN_{\text{cov}}^r\), and \(CN_{\text{cov}}^B = CN_{\text{cov}}^s\).

\vspace{10pt}
\noindent
The Gaussian weighting for each reference system is given by:
\begin{equation}
  W_A^r(CN_{cov}^A, CN_{cov}^{A,r}) = \sum_{j=1}^{N_{gauss}} \frac{1}{\mathcal{N}} \exp\left[-6j \cdot (CN_{cov}^A - CN_{cov}^{A,r})^2\right]
\end{equation}

with
\begin{equation}
  \sum_{r}^{N_{A,ref}} W_A^r(CN_{cov}^A, CN_{cov}^{A,r}) = 1
\end{equation}

\vspace{10pt}
\noindent
\(\mathcal{N}\) is a normalization constant.

\begin{equation}
  \mathcal{N} = \sum_{A,\text{ref}=1}^{N^{A,\text{ref}}} \exp\left[-6j \cdot (CN^A - CN^{A,\text{ref}})^2\right]
\end{equation}

// Write r or ref? CN with or without cov?

\vspace{10pt}
\noindent
The number of Gaussian function per reference system \(N_{gauss}\) is mostly one, but equal to three for \(CN_{cov}^{A,r} = 0\) and reference systems with similar coordination number.


\vspace{10pt}
\noindent
\(C_6^{AB}\) is the pairwise dipole-dipole dispersion coefficients calculated by numerical integration via the Casimir-Polder relation.
\begin{equation}
  C_6^{AB} = \frac{3}{\pi} \sum_{j} w_j \overline{\alpha}_A (i\omega_j, q_A, CN_{cov}^A)\overline{\alpha}_B (i\omega_j, q_B, CN_{cov}^B)
\end{equation}

\noindent
\(w_j\) are the integration weights, which are derived from a trapeziodal partitioning between the grid points \(j \in \{2, \dots, 22\}\).


\noindent
The isotropically averaged, dynamic dipole-dipole polarizabilites \(\overline{\alpha}\) at the \(j\)th imaginary frequency \(i\omega_j\) are obtained from the self-consistent D4 model; i.e., they are depending on the covalent coordination number and are also charge dependent.

\begin{equation}
  \overline{\alpha}_A(i\omega_j, q_A, CN_{cov}^A) = \sum_{r}^{N_{A,ref}} \xi_A^r (q_A, q_{A,r}) \overline{\alpha}_{A,r}(i\omega_j, q_{A,r}, CN_{cov}^{A,r}) W_A^r(CN_{cov}^A, CN_{cov}^{A,r})
\end{equation}

%\begin{equation}
%  \overline{\alpha}_{A,r}(i\omega_j, q_{A,r}, CN_{cov}^{A,r}) = \sum_{A,\text{ref}=1}^{N^{A,\text{ref}}} \alpha^{A,\text{ref}} (i\omega, z^A) W_A^{A,\text{ref}}
%\end{equation}
\begin{equation}
  \overline{\alpha}_{A,r}(i\omega_j, q_{A,r}, CN_{cov}^{A,r}) = \sum_{A,\text{ref}=1}^{N^{A,\text{ref}}} \alpha^{A,\text{ref}} (i\omega, z^A) W_A^r
\end{equation}

%\begin{equation} \label{eq:44}
%  W_A^{A,\text{ref}}(CN^A, CN^{A,\text{ref}}) = \frac{\sum_{j=1}^{N^s} exp\left(-\beta_2 \times j(CN^A - CN^{A,\text{ref}})^2\right)}{\sum_{A,\text{ref}=1}^{N^{A,\text{ref}}} \sum_{j=1}^{N^s} \exp\left(-\beta_2 \times j(CN^A - CN^{A,\text{ref}})^2\right)}
%\end{equation}
%
%\noindent
%The parameter \(\beta_2 = 6\) is adjusted manually to guarantee a smooth weighting function. // Is \(\beta_2\) just 6?
%
%\noindent
%\(N^s\) is determined at runtime for every reference system.

%\noindent
%Differences between CN values of reference systems are used to indicate when another more compact Gaussian is necessary.
%
%\noindent
%Reference systems with a sufficiently large CN difference can easily be distinguished from each other within the Gaussian weighting procedure in equation \ref{eq:44} for \(N^s\) = 1.
%
%\noindent
%This might not be true for smaller CN differences.
%The set of Gaussian functions is enlarged by increasing \(N^s\) in order to add less overlapping functions, which makes the reference systems distinguishable within the Gaussian weighting procedure in equation \ref{eq:44}.

\noindent
The charge-dependent atomic dynamic polarizability for a single reference system of atom A is given by the product of \(\alpha^{A,\text{ref}}(i\omega)\) and its scaling function as:

%\begin{equation} \label{eq:44}
%  \alpha^{A,\text{ref}}(i\omega, z^A) = \alpha^{A,\text{ref}}(i\omega) \zeta(z^A, z^{A,\text{ref}})
%\end{equation}
\begin{equation}
  \alpha^{A,\text{ref}}(i\omega, q_A) = \alpha^{A,\text{ref}}(i\omega) \xi_A^r(q_A, q_{A,r})
\end{equation}


%\begin{equation} \label{eq:46}
%  \alpha^{A,\text{ref}}(i\omega) = \frac{1}{m} \left[\alpha^{AmXn}(i\omega) - \frac{n}{l} \alpha^{X_l}(i\omega) \zeta(z^X, z^{X,\text{ref}})\right]
%\end{equation}
\begin{equation} \label{eq:46}
  \alpha^{A,\text{ref}}(i\omega) = \frac{1}{m} \left[\alpha^{AmXn}(i\omega) - \frac{n}{l} \alpha^{X_l}(i\omega) \xi_A^r(q_X, q_{X,r})\right]
\end{equation}

// The effective nuclear charges \(z^{X,\text{ref}}\) entering equation \ref{eq:46} are constant values determined once for the respective reference system. (Find out how to get them)

%\noindent
%An example of the notation for a CN difference within a reference system is \(\Delta CN_{23}\), which indicates the CN difference between reference system A,ref2 and A,ref3.


\vspace{10pt}
\noindent
The charge-dependency is included via the empirical scaling function \(\xi_A^r\).
\begin{equation}
  \xi_A^r(q_A, q_{A,r}) = \exp\left[3\left\{1-\exp\left[4\eta_A\left(1-\frac{Z_A^{eff} + q_{A,r}}{Z_A^{eff} + q_A}\right)\right]\right\}\right]
\end{equation}

\noindent
where \(\eta_A\) is the chemical hardness taken from ref 98.

\noindent
\(Z_A^{eff}\) is the effective nuclear charge of atom A.



\vspace{10pt}
\noindent
\(C_8^{AB}\) is calculated recursively from the lowest order \(C_6^{AB}\) coefficients.

\begin{equation}
  C_8^{AB} = 3C_6^{AB} \sqrt{\mathcal{Q}^A\mathcal{Q}^B}
\end{equation}

\begin{equation}
  \mathcal{Q}^A = s_{42} \sqrt{Z^A} \frac{\braket{r^4}^A}{\braket{r^2}^A}
\end{equation}


\vspace{10pt}
\noindent
\(\sqrt{Z^A}\) is the ad hoc nuclear charge dependent factor.

From the original xTB program we can see that \(s_{42}\) is \(0.5\), and \(Z^A\) is the atomic number of A.

\begin{equation}
  \sqrt{0.5 \left(\frac{r^4}{r^2} \sqrt{Z^A}\right)}
\end{equation}

\(\braket{r^4}\) and \(\braket{r^2}\) are simple multipole-type expectation values derived from atomic densities which are averaged geometrically to get the pair coefficients.

\vspace{20pt}
\noindent
\(CN^A_{cov}\) is the covalent coordination number for atom A.

\vspace{10pt}
\noindent
\(q\) is the atomic charge, so \(q_A\) is the atomic charge for atom A.

\vspace{10pt}
\noindent
The scaling parameters in the dispersion model are:
\[
  a1 = 0.52 \quad|\quad a2 = 5.0 \quad|\quad s6 = 1.0 \quad|\quad s8 = 2.7
\]

\vspace{10pt}
\noindent
BJ = Becke-Johnson

\begin{equation}
  f_n^{damp,BJ}(R_{AB}) = \frac{R_{AB}^n}{R_{AB}^n + (a_1 \times R_{AB}^{crit} + a_2)^6}
\end{equation}

\begin{equation}
  R_{AB}^{crit} = \sqrt{\frac{C_8^{AB}}{C_6^{AB}}}
\end{equation}


\begin{equation}
  f_9^{damp,zero}(R_{AB}, R_{AC}, R_{BC}) = \left(1 + 6 \left(\sqrt{\frac{R_{AB}^{crit} R_{BC}^{crit} R_{CA}^{crit}}{R_{AB} R_{BC} R_{CA}}}\right)^{16}\right)^{-1}
\end{equation}
