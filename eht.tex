\section{Extended Hückel Theory Matrix for GFN2-xTB}
\begin{equation}
\begin{split}
    \mn{H^{EHT}} &= \frac{1}{2}K^{ll'}_{AB}\mn{S}(H_{\mu\mu}+H_{\nu\nu})\\&\cdot X(EN_A,EN_B)\\&\cdot \Pi(R_{AB},l,l')\\&\cdot Y(\zeta^A_l,\zeta^B_{l'}), \forall \mu \in l(A), \nu \in l'(B)
\end{split}
\end{equation}
where $\mu$ and $\nu$ are AO indecies, $l$ and $l'$ index shells. Both AO's are associated with an atom labled A and B. 
$K^{ll'}_{AB}$ is a element and shell specific fitted constant however, in GFN2 it only depends on the shells. 
$S_{\mu\nu}=\braket{\phi_\mu|\phi_\nu}$ is just the overlap of the orbitals. In GFN2 $H_{\kappa\kappa}=h^l_A-\delta h^l_{CN'_A}CN'_A$ where $CN'_A$ is the modified GFN2-type Coordinate Number for the element of atom A.

\begin{align}
\begin{split}
  CN^{'}_A = &\sum^{N_\text{atoms}}_{B \neq A} (1 + e^{-10(4(R_{A,\text{cov}} + R_{B,\text{cov}})/3R_{AB}-1)})^{-1} \\
             &\times (1 + e^{-20(4(R_{A,\text{cov}} + R_{B,\text{cov}} + 2)/3R_{AB}-1)})^{-1}
\end{split}
\end{align}

$h^l_A$ and $\delta h^l_{CN'_A}$ are both fitted constants. $EN_A$ is the electronegativity of the element of atom A, given in the original \texttt{xtb} code. 

\begin{align}
    X(EN_A,EN_B) &= 1 + k_{EN}\Delta EN_{AB}^2\\
    k_{EN} &= 0.02 \text{ in GFN2}\\
    \Delta EN_{AB}^2 &= (EN_A-EN_B)^2  
\end{align}
%The electronegativity for C and H are 2.55 and 2.20 according to wikipedia.
%Thus here is a table for the combinations we will be working with:\\ 
%\begin{tabular}{c|c|l}
%    A&B&$X(EN_A,EN_B)$\\
%    \hline
%    C&C&$1$\\
%    C&H&$1+0.02\cdot (0.35^2)$\\
%    H&C&$1+0.02\cdot (0.35^2)$\\
%    H&H&$1$\\
%\end{tabular}
\begin{equation}
\begin{split}
    \Pi(R_{AB},l,l') &= \left(1 + k^{\text{poly}}_{A,l}\left(\frac{R_{AB}}{R_{\text{cov},AB}}\right)^\frac{1}{2}\right)\left(1 + k^{\text{poly}}_{B,l'}\left(\frac{R_{AB}}{R_{\text{cov},AB}}\right)^\frac{1}{2}\right)\\
\end{split}
\end{equation}
$R_{\text{cov},AB}$ are the summed covalent radii (\(R_{\text{cov},A} + R_{\text{cov},B}\)), e.g. $R_{\text{cov},H}=0.32$, $R_{\text{cov},C}=0.75$ are given in the original \texttt{xtb} code. $k^{\text{poly}}_{A,l}$ and $k^{\text{poly}}_{B,l'}$ are element and shell specific constants. 
\begin{equation}
\begin{split}
    Y(\zeta^A_l,\zeta^B_{l'}) &= \left(\frac{2\sqrt{\zeta^A_l\zeta^B_{l'}}}{\zeta^A_l+\zeta^B_{l'}}\right)^\frac{1}{2}\\
\end{split}
\end{equation}
Here, $\zeta^A_l$ are the STO exponents of the GFN2-xTB AO basis.\\
Slater Type Orbitals are defined as such: 
\begin{equation}
\chi_{\zeta,n,l,m}(r, \theta, \varphi) = NY_{l,m}(\theta, \varphi)r^{n-1}e^{-\zeta r}
\end{equation}
N is a normalisation constant, Y are spherical harmonic funtions, n, l, m are the quantum numbers for the AO. $r,\theta,\varphi$ are polar 3D coordinates. $\zeta$ determines the radial extent of the STO, a large value gives rise to a function that is "tight" around the nucleus and a small value gives a more "diffuse" function. This $\zeta$ is the one mentioned in the Y term of $\E{EHT}$ and is a value fitted when constructing the basis set, thus it is given to us.  

